\documentclass{article}
\usepackage{noweb}
\usepackage{amsmath}
\usepackage{fancyvrb}
\addtolength{\textwidth}{1in}
\addtolength{\oddsidemargin}{-.5in}
\setlength{\evensidemargin}{\oddsidemargin}

\newcommand{\myfig}[1]{\resizebox{\textwidth}{!}
                        {\includegraphics{figure/#1.pdf}}}
\newcommand{\code}[1]{\texttt{#1}}
\title{The \emph{pedigree} functions in R}
\author{Terry Therneau and Elizabeth Atkinson}

\begin{document}
\maketitle
\tableofcontents
\section{Introduction}
The pedigree routines came out of a simple need -- to quickly draw a
pedigree structure on the screen, within R, that was ``good enough'' to
help with debugging the actual routines of interest, which were those for
fitting mixed effecs Cox models to large family data.  As such the routine
had compactness and automation as primary goals; complete annotation
(monozygous twins, multiple types of affected status) and most certainly
elegance were not on the list.  Other software could do that much
better.

It therefore came as a major surprise when these routines proved useful
to others.  Through their constant feedback, application to more
complex pedigrees, and ongoing requests for one more feature, the routine has 
become what it is today.  This routine is still not 
suitable for really large pedigrees, nor for heavily inbred ones such as in
animal studies, and will likely not evolve in that way.  The authors' fondest%'
hope is that others will pick up the project.

\section{Pedigree}
The pedigree function is the first step, creating an object of class
\emph{pedigree}.  
It accepts the following input
\begin{description}
  \item[id] A numeric or character vector of subject identifiers.
  \item[dadid] The identifier of the father.
  \item[momid] The identifier of the mother.
  \item[sex] The gender of the individual.  This can be a numeric variable
    with codes of 1=male, 2=female, 3=unknown, 4=terminated, or NA=unknown.
    A character or factor variable can also be supplied containing
    the above; the string may be truncated and of arbitrary case.  A sex
    value of 0=male 1=female is also accepted.
  \item[status] Optional, a numeric variable with 0 = censored and 1 = dead.
  \item[relationship] Optional, a matrix or data frame with three columns.
    The first two contain the identifier values of the subject pairs, and
    the third the code for their relationship:
    1 = Monozygotic twin, 2= Dizygotic twin, 3= Twin of unknown zygosity,
    4 = Spouse.  
  \item[famid] Optional, a numeric or character vector of family identifiers.
\end{description}

The \Verb!famid! variable is placed last as it was a later addition to the
code; thus prior invocations of the function that use positional 
arguments won't be affected.                                       %'
If present, this allows a set of pedigrees to be generated, one per
family.  The resultant structure will be an object of class
\Verb!pedigreeList!.

Note that a factor variable is not listed as one of the choices for the
subject identifier. This is on purpose.  Factors
were designed to accomodate character strings whose values came from a limited
class -- things like race or gender, and are not appropriate for a subject
identifier.  All of their special properties as compared to a character
variable turn out to be backwards for this case, in particular a memory
of the original level set when subscripting is done.
However, due to the awful decision early on in S to automatically turn every
character into a factor --- unless you stood at the door with a club to
head the package off --- most users have become ingrained to the idea of
using them for every character variable. 
(I encourage you to set the global option stringsAsFactors=FALSE to turn
off autoconversion -- it will measurably improve your R experience).
Therefore, to avoid unnecessary hassle for our users 
the code will accept a factor as input for the id variables, but
the final structure does not retain it.  
Gender and relation do become factors.  Status follows the pattern of the 
survival routines and remains an integer.

We will describe the code in a set of blocks.
\begin{nwchunk}
\nwhypn{pedigree}=
 pedigree <- function(id, dadid, momid, sex, affected, status, relation,
                      famid, missid) \{
     \nwhypf{pedigree-error1}{pedigree-error}{pedigree-error2}
     \nwhypf{pedigree-parent1}{pedigree-parent}{pedigree-parent2}
     \nwhypf{pedigree-create1}{pedigree-create}{pedigree-create2}
     \nwhypf{pedigree-extra1}{pedigree-extra}{pedigree-extra2}
     if (missing(famid)) class(temp) <- 'pedigree'
     else class(temp) <- 'pedigreeList'
     temp
     \}
 \nwhypf{pedigree-subscript1}{pedigree-subscript}{pedigree-subscript2}
\end{nwchunk}

\subsection{Data checks}
The code starts out with some checks on the input data.  
Is it all the same length, are the codes legal, etc.
\begin{nwchunk}
\nwhyp{pedigree-error2}{pedigree-error}{pedigree-error1}{pedigree-error3}=
 n <- length(id)
 if (length(momid) != n) stop("Mismatched lengths, id and momid")
 if (length(dadid) != n) stop("Mismatched lengths, id and momid")
 if (length(sex  ) != n) stop("Mismatched lengths, id and sex")
 
 # Don't allow missing id values
 if (any(is.na(id))) stop("Missing value for the id variable")
 if (!is.numeric(id)) \{
     id <- as.character(id)
     if (length(grep('^ *$', id)) > 0)
     stop("A blank or empty string is not allowed as the id variable")
   \}
 
 # Allow for character/numeric/factor in the sex variable
 if(is.factor(sex))
         sex <- as.character(sex)
 codes <- c("male","female", "unknown", "terminated")
 if(is.character(sex)) sex<- charmatch(casefold(sex, upper = FALSE), codes, 
                                       nomatch = 3)        
 
 # assume either 0/1/2/4 =  female/male/unknown/term, or 1/2/3/4
 #  if only 1/2 assume no unknowns
 if(min(sex) == 0)
         sex <- sex + 1
 sex <- ifelse(sex < 1 | sex > 4, 3, sex)
 if(all(sex > 2))
         stop("Invalid values for 'sex'")
     else if(mean(sex == 3) > 0.25)
             warning("More than 25% of the gender values are 'unknown'")
 sex <- factor(sex, 1:4, labels = codes)
\end{nwchunk}

Create the variables descibing a missing father and/or mother,
which is what we expect both for people at the top of the
pedigree and for marry-ins, \emph{before} adding in the family
id information.  
It's easier to do it first.
If there are multiple families in the pedigree, make a working set of
identifiers that are of the form `family/subject'.
Family identifiers can be factor, character, or numeric.
\begin{nwchunk}
\nwhypb{pedigree-error3}{pedigree-error}{pedigree-error2}=
 if (missing(missid)) \{
     if (is.numeric(id)) missid <- 0
     else missid <- ""
 \}
 
 nofather <- (is.na(dadid) | dadid==missid)
 nomother <- (is.na(momid) | momid==missid)
 
 if (!missing(famid)) \{
     if (any(is.na(famid))) stop("The family id cannot contain missing values")
     if (is.factor(famid) || is.character(famid)) \{
         if (length(grep('^ *$', famid)) > 0)
             stop("The family id cannot be a blank or empty string")
         \}
     #Make a temporary new id from the family and subject pair
     oldid <-id
     id <- paste(as.character(famid), as.character(id), sep='/')
     dadid <- paste(as.character(famid), as.character(dadid), sep='/')
     momid <- paste(as.character(famid), as.character(momid), sep='/')
     \}
 
 if (any(duplicated(id))) \{
     duplist <- id[duplicated(id)]
     msg.n <- min(length(duplist), 6)
     stop(paste("Duplicate subject id:", duplist[1:msg.n]))
     \}
\end{nwchunk}

Next check that any mother or father identifiers are found in the identifier
list, and are of the right sex.
Subjects who don't have a mother or father are founders.  For those people %'
both of the parents should be missing.

\begin{nwchunk}
\nwhypb{pedigree-parent2}{pedigree-parent}{pedigree-parent1}=
 findex <- match(dadid, id, nomatch = 0)
 if(any(sex[findex] != "male")) \{
     who <- unique((id[findex])[sex[findex] != "male"])
     msg.n <- 1:min(5, length(who))  #Don't list a zillion
     stop(paste("Id not male, but is a father:", 
                paste(who[msg.n], collapse= " ")))
     \}
 
 if (any(findex==0 & !nofather)) \{
     who <- dadid[which(findex==0 & !nofather)]
     msg.n <- 1:min(5, length(who))  #Don't list a zillion
     stop(paste("Value of 'dadid' not found in the id list", 
                paste(who[msg.n], collapse= " ")))
     \}
     
 mindex <- match(momid, id, nomatch = 0)
 if(any(sex[mindex] != "female")) \{
     who <- unique((id[mindex])[sex[mindex] != "female"])
     msg.n <- 1:min(5, length(who))
     stop(paste("Id not female, but is a mother:", 
                paste(who[msg.n], collapse = " ")))
     \}
 
 if (any(mindex==0 & !nomother)) \{
     who <- momid[which(mindex==0 & !nomother)]
     msg.n <- 1:min(5, length(who))  #Don't list a zillion
     stop(paste("Value of 'momid' not found in the id list", 
                paste(who[msg.n], collapse= " ")))
     \}
 
 if (any(mindex==0 & findex!=0) || any(mindex!=0 & findex==0)) \{
     who <- id[which((mindex==0 & findex!=0) |(mindex!=0 & findex==0))] 
     msg.n <- 1:min(5, length(who))  #Don't list a zillion
     stop(paste("Subjects must have both a father and mother, or have neither",
                paste(who[msg.n], collapse= " ")))
 \}
 
 if (!missing(famid)) \{
     if (any(famid[mindex] != famid[mindex>0])) \{
         who <- (id[mindex>0])[famid[mindex] != famid[mindex>0]]
         msg.n <- 1:min(5, length(who))
         stop(paste("Mother's family != subject's family", 
                    paste(who[msg.n], collapse=" ")))
         \}
     if (any(famid[findex] != famid[findex>0])) \{
         who <- (id[findex>0])[famid[findex] != famid[findex>0]]
         msg.n <- 1:min(5, length(who))
         stop(paste("Father's family != subject's family", 
                    paste(who[msg.n], collapse=" ")))
         \}
     \}
\end{nwchunk}

\subsection{Creation}
Now, paste the parts together into a basic pedigree.
The fields for father and mother are not the identifiers of
the parents, but their row number in the structure.
\begin{nwchunk}
\nwhypb{pedigree-create2}{pedigree-create}{pedigree-create1}=
 if (missing(famid))
     temp <- list(id = id, findex=findex, mindex=mindex, sex=sex)
 else temp<- list(famid=famid, id=oldid, findex=findex, mindex=mindex, 
                  sex=sex)
\end{nwchunk}

The last part is to check out the optional features,
affected status, survival status, and relationships.

Update by Jason Sinnwell, 5/2011: Allow missing values (NA) in the 
affected status matrix. 

Update by Jason Sinnwell 7/2011: Change relation:id1 and id2 to indx1 and indx2
because they are the index of the id vector. Both $pedigree.trim$ 
and $[.pedigree$ now work with these column names.

\begin{nwchunk}
\nwhypb{pedigree-extra2}{pedigree-extra}{pedigree-extra1}=
 if (!missing(affected)) \{
     if (is.matrix(affected))\{
         if (nrow(affected) != n) stop("Wrong number of rows in affected")
         if (is.logical(affected)) affected <- 1* affected
         \} 
     else \{
         if (length(affected) != n)
             stop("Wrong length for affected")
 
         if (is.logical(affected)) affected <- as.numeric(affected)
         if (is.factor(affected))  affected <- as.numeric(affected) -1
         \}
     if(max(affected, na.rm=TRUE) > min(affected, na.rm=TRUE)) 
       affected <- affected - min(affected, na.rm=TRUE)
     if (!all(affected==0 | affected==1 | is.na(affected)))
                 stop("Invalid code for affected status")
     temp$affected <- affected
     \}
 
 if(!missing(status)) \{
     if(length(status) != n)
         stop("Wrong length for affected")
     if (is.logical(status)) status <- as.integer(status)
     if(any(status != 0 & status != 1))
         stop("Invalid status code")
     temp$status <- status
     \}
 
 if (!missing(relation)) \{
     if (!missing(famid)) \{
         if (is.matrix(relation)) \{
             if (ncol(relation) != 4) 
                 stop("Relation matrix must have 3 columns + famid")
             id1 <- relation[,1]
             id2 <- relation[,2]
             code <- relation[,3]
             famid <- relation[,4]
             \}
         else if (is.data.frame(relation)) \{
             id1 <- relation$id1
             id2 <- relation$id2
             code <- relation$code
             famid <- relation$famid
             if (is.null(id1) || is.null(id2) || is.null(code) ||is.null(famid)) 
             stop("Relation data must have id1, id2, family id and code")
             \}
         else stop("Relation argument must be a matrix or a dataframe")
         \}
     else \{
         if (is.matrix(relation)) \{
             if (ncol(relation) != 3) 
                 stop("Relation matrix must have 3 columns")
             id1 <- relation[,1]
             id2 <- relation[,2]
             code <- relation[,3]
             \}
         else if (is.data.frame(relation)) \{
             id1 <- relation$id1
             id2 <- relation$id2
             code <- relation$code
             if (is.null(id1) || is.null(id2) || is.null(code)) 
                 stop("Relation data frame must have id1, id2, and code")
             \}
         else stop("Relation argument must be a matrix or a list")
         \}
     
     if (!is.numeric(code))
         code <- match(code, c("MZ twin", "DZ twin", "UZ twin", "spouse"))
     else code <- factor(code, levels=1:4,
                         labels=c("MZ twin", "DZ twin", "UZ twin", "spouse"))
     if (any(is.na(code)))
         stop("Invalid relationship code")
      
     # Is everyone in this relationship in the pedigree?
     if (!missing(famid)) \{
         temp1 <- match(paste(as.character(famid), as.character(id1), sep='/'), 
                        id, nomatch=0)
         temp2 <- match(paste(as.character(famid), as.character(id2), sep='/'),
                        id, nomatch=0)
       \}
     else \{
         temp1 <- match(id1, id, nomatch=0)
         temp2 <- match(id2, id, nomatch=0)
       \}
     
     if (any(temp1==0 | temp2==0))
         stop("Subjects in relationships that are not in the pedigree")
     if (any(temp1==temp2)) \{
         who <- temp1[temp1==temp2]
         stop(paste("Subject", id[who], "is their own spouse or twin"))
         \}
 
     # Check, are the twins really twins?
     ncode <- as.numeric(code)
     if (any(ncode<3)) \{
         twins <- (ncode<3)
         if (any(momid[temp1[twins]] != momid[temp2[twins]]))
             stop("Twins found with different mothers")
         if (any(dadid[temp1[twins]] != dadid[temp2[twins]]))
             stop("Twins found with different fathers")
         \}
     # Check, are the monozygote twins the same gender?
     if (any(code=="MZ twin")) \{
         mztwins <- (code=="MZ twin")
         if (any(sex[temp1[mztwins]] != sex[temp2[mztwins]]))
             stop("MZ Twins with different genders")
         \}
 
     ##Use id index as indx1 and indx2
     if (!missing(famid)) \{
         temp$relation <- data.frame(famid=famid, indx1=temp1, indx2=temp2, code=code)
         
       \}
     else temp$relation <- data.frame(indx1=temp1, indx2=temp2, code=code)
     \}
\end{nwchunk}

The final structure will be in the order of the original data, and all the
components except \Verb!relation! will have the
same number of rows as the original data.


\subsection{Subscripting}

Subscripting of a pedigree list extracts one or more families from the
list.  We treat character subscripts in the same way that dimnames on
a matrix are used.  Factors are a problem though: assume that we
have a vector x with names ``joe'', ``charlie'', ``fred'', then
\Verb!x['joe']! is the first element of the vector, but
\Verb!temp <- factor('joe', 'charlie', 'fred'); z <- temp[1]; x[z] ! will
be the second element! 
R is implicitly using as.numeric on factors when they are a subscript;
this caught an early version of the code when an element of a data
frame was used to index the pedigree: characters are turned into factors
when bundled into a data frame.

Note:
\begin{enumerate}
  \item What should we do if the family id is a numeric: when the user
    says [4] do they mean the fourth family in the list or family '4'?
    The user is responsible to say ['4'] in this case.
  \item  In a normal vector invalid subscripts give an NA, e.g. (1:3)[6], but
    since there is no such object as an ``NA pedigree'', we emit an error
    for this.
  \item The \Verb!drop! argument has no meaning for pedigrees, but must to be
    a defined argument of any subscript method; we simply ignore it.
  \item Updating the father/mother is a minor nuisance;
    since they must are integer indices to rows they must be
    recreated after selection.  Ditto for the relationship matrix.  
\end{enumerate}
\begin{nwchunk}
\nwhyp{pedigree-subscript2}{pedigree-subscript}{pedigree-subscript1}{pedigree-subscript3}=
 "[.pedigreeList" <- function(x, ..., drop=F) \{
     if (length(list(...)) != 1) stop ("Only 1 subscript allowed")
     ufam <- unique(x$famid)
     if (is.factor(..1) || is.character(..1)) indx <- ufam[match(..1, ufam)]
     else indx <- ufam[..1]
         
     if (any(is.na(indx))) 
             stop(paste("Familiy", (..1[is.na(indx)])[1], "not found"))
 
     keep <- which(x$famid %in% indx)  #which rows to keep
     for (i in c('id', 'famid', 'sex'))
         x[[i]] <- (x[[i]])[keep]
     
     kept.rows <- (1:length(x$findex))[keep]
     x$findex <- match(x$findex[keep], kept.rows, nomatch=0)
     x$mindex <- match(x$mindex[keep], kept.rows, nomatch=0)
     
     #optional components
     if (!is.null(x$status)) x$status <- x$status[keep]
     if (!is.null(x$affected)) \{
         if (is.matrix(x$affected)) x$affected <- x$affected[keep,,drop=FALSE]
         else x$affected <- x$affected[keep]
         \}
     if (!is.null(x$relation)) \{
         keep <- !is.na(match(x$relation$famid, indx))
        if (any(keep)) \{
             x$relation <- x$relation[keep,,drop=FALSE]
             ##Update twin id indexes
             x$relation$indx1 <- match(x$relation$indx1, kept.rows, nomatch=0)
             x$relation$indx2 <- match(x$relation$indx2, kept.rows, nomatch=0)
             ##If only one family chosen, remove famid
             if (length(indx)==1) \{x$relation$famid <- NULL\}
             \}
         else x$relation <- NULL  # No relations matrix elements for this family
         \}
     
     if (length(indx)==1)  class(x) <- 'pedigree'  #only one family chosen
     else class(x) <- 'pedigreeList'
     x
     \}
\end{nwchunk}

For a pedigree, the subscript operator extracts a subset of individuals.
We disallow selections that retain only 1 of a subject's parents, since    %'
they cause plotting trouble later.
Relations are worth keeping only if both parties in the relation were
selected.

\begin{nwchunk}
\nwhypb{pedigree-subscript3}{pedigree-subscript}{pedigree-subscript2}=
 "[.pedigree" <- function(x, ..., drop=F) \{
     if (length(list(...)) != 1) stop ("Only 1 subscript allowed")
     if (is.character(..1) || is.factor(..1)) i <- match(..1, x$id)
     else i <- (1:length(x$id))[..1]
     
     if (any(is.na(i))) paste("Subject", ..1[which(is.na(i))][1], "not found")
 
     z <- list(id=x$id[i],findex=match(x$findex[i], i, nomatch=0),
               mindex=match(x$mindex[i], i, nomatch=0),
              sex=x$sex[i])
     if (!is.null(x$affected)) \{
         if (is.matrix(x$affected)) z$affected <- x$affected[i,, drop=F]
         else z$affected <- x$affected[i]
     \}
     if (!is.null(x$famid)) z$famid <- x$famid[i]
 
    
     if (!is.null(x$relation)) \{
       indx1 <- match(x$relation$indx1, i, nomatch=0)
       indx2 <- match(x$relation$indx2, i, nomatch=0)
       keep <- (indx1 >0 & indx2 >0)  #keep only if both id's are kept
       if (any(keep)) \{
         z$relation <- x$relation[keep,,drop=FALSE]
         z$relation$indx1 <- indx1[keep]
         z$relation$indx2 <- indx2[keep]
       \}
     \}
     
     if (!is.null(x$hints)) \{
         temp <- list(order= x$hints$order[i])
         if (!is.null(x$hints$spouse)) \{
             indx1 <- match(x$hints$spouse[,1], i, nomatch=0)
             indx2 <- match(x$hints$spouse[,2], i, nomatch=0)
             keep <- (indx1 >0 & indx2 >0)  #keep only if both id's are kept
             if (any(keep))
                 temp$spouse <- cbind(indx1[keep], indx2[keep],
                                      x$hints$spouse[keep,3])
             \}
         z$hints <- temp
         \}
 
     if (any(z$findex==0 & z$mindex>0) | any(z$findex>0 & z$mindex==0))
         stop("A subpedigree cannot choose only one parent of a subject")
     class(z) <- 'pedigree'
     z
     \}
\end{nwchunk}

\subsection{As Data.Frame}

Convert the pedigree to a data.frame so it is easy to view when removing or
trimming individuals with their various indicators.  
The relation and hints elements of the pedigree object are not easy to
put in a data.frame with one entry per subject. These items are one entry 
per subject, so are put in the returned data.frame:  id, findex, mindex, 
sex, affected, status.  The findex and mindex are converted to the actual id
of the parents rather than the index.

Can be used with as.data.frame(ped) or data.frame(ped). Specify in Namespace
file that it is an S3 method.



\begin{nwchunk}
\nwhypn{as.data.frame.pedigree}=
 
 as.data.frame.pedigree <- function(x, ...) \{
 
   dadid <- momid <- rep(0, length(x$id))
   dadid[x$findex>0] <- x$id[x$findex]
   momid[x$mindex>0] <- x$id[x$mindex]
   df <- data.frame(id=x$id, dadid=dadid, momid=momid, sex=x$sex)
   
   if(!is.null(x$affected))
     df$affected = x$affected
   
   if(!is.null(x$status))
     df$status = x$status
   return(df)
 \}
\end{nwchunk}


This function is useful for checking the pedigree object with the
$findex$ and $mindex$ vector instead of them replaced with the ids of 
the parents.  This is not currently included in the package.

\begin{nwchunk}
\nwhypn{ped2df}=
 
 ped2df <- function(ped) \{
   df <- data.frame(id=ped$id, findex=ped$findex, mindex=ped$mindex, sex=ped$sex)
   if(!is.null(ped$affected))
     df$affected = ped$affected
 
   if(!is.null(ped$status))
     df$status = ped$status
 
   return(df)
 
 \}
 
\end{nwchunk}



\subsection{Printing}
It usually doesn't make sense to print a pedigree, since the id is just   %'
a repeat of the input data and the family connections are pointers.
Thus we create a simple summary.

\begin{nwchunk}
\nwhypn{print.pedigree}=
 print.pedigree <- function(x, ...) \{
     cat("Pedigree object with", length(x$id), "subjects")
     if (!is.null(x$famid)) cat(", family id=", x$famid[1], "{\textbackslash}n")
     else cat("{\textbackslash}n")
     cat("Bit size=", bitSize(x)$bitSize, "{\textbackslash}n")
     \}
 
 print.pedigreeList <- function(x, ...) \{
     cat("Pedigree list with", length(x$id), "total subjects in",
         length(unique(x$famid)), "families{\textbackslash}n")
     \}
\end{nwchunk}
\section{Kinship matrices}
The kinship matrix is foundational for random effects models with family
data.  
For $n$ subjects it is an $n \times n$ matrix whose $ij$ element contains
the expected fraction of alleles that would be identical by descent
if we sampled one from subject $i$ and another from subject $j$.
Note that the diagonal elements of the matrix will be 0.5 not 1: when we
randomly sample twice from the same subject (with replacement) 
we will get two copies of the gene inherited from the father 1/4 of the
time, the maternal copy twice (1/4) or one of each 1/2 the time.
The formal definition is $K(i,i) = 1/4 + 1/4 + 1/2 K(m,f)$ where
$m$ and $f$ are the father and mother of subject $i$.

The algorithm used is found in K Lange, 
\emph{Mathematical and Statistical  Methods for Genetic Analysis}, 
Springer 1997, page 71--72.

The key idea of the recursive algorithm for $K(i,j)$ is to condition on
the gene selection for the first index $i$.
Let $m(i)$ and $f(i)$ be the indices of the mother and father of subject $i$
and $g$ be the allele randomly sampled from subject $i$,
which may of either maternal or paternal origin.

% updated to haveby JPS, 4/15/13

\begin{align}
  K(i,j) &= P(\mbox{$g$ maternal}) * K(m(i), j) + 
            P(\mbox{$g$ paternal}) * K(f(i), j) \label{recur0} \\
         &= 1/2 K(m(i), j) + 1/2 K(f(i), j)   \label{recur1} \\
  K(i,i) &= 1/2(1 + K(m(i), f(i))) \label{self} 
\end{align}

The key step in equation \eqref{recur0} is if $g$ has a maternal origin, then
it is a random selection from the two maternal genes, and it's IBD state with
respect to subject $j$ is that of a random selection from m(i) to a random
selection from $j$.  This is precisely the definition of $K(m(i), j)$.
The recursion does not work for $K(i,i)$ in equation \eqref{self} since once 
we select a maternal gene the second choice from ``$j$'' cannot use a 
different maternal gene.


For the recurrence algorithm to work properly we need to compute the
values of $K$ for any parent before the calculations for their children.
Pedigree founders (those with no parents) are assumed to be unassociated,
so for these subjects we have
\begin{align*}
  K(i,i) &= 1/2
  K(i,j) &=0 \; i\ne j
\end{align*}

The final formula slightly different for the $X$ chromosome. 
Equation \ref{recur0} still holds, but for males the probability
that a selected $X$ chromosome is maternal is 1, so when $i$ a male
the recurrence formula becomes $K(i,j) = K(m(i),j)$.  
For females it is unchanged.
All males will have $K(i,i) = 1$ for the $X$ chromosome.

In order to have already-defined terms on the right hand side of the
recurrence formula for each element, subjects need to be processed
in the following order
\begin{enumerate}
  \item Generation 0 (founders)
  \item $K(i,j)$ where $i$ is from generation 1 and $j$ from generation 0.
  \item $K(i,j)$ with $i$ and $j$ from generation 1
  \item $K(i,j)$ with $i$ from generation 2 and $j$ from generation 0 or 1
  \item $K(i,j)$ with $i$ and $j$ from generation 2.
  \item \ldots
\end{enumerate}
The kindepth routine assigns a plotting depth to each subject in such
a way that parents are always above children.  
For each depth we need to do the compuations of formula \eqref{recur}
twice.  The first time it will get the relationship between each subject
and prior generations correct, the second will correctly compute the
values between subjects on the same level.
The computations within any stage of the above list can be vectorized,
but not those between stages.

Let \Verb!indx! be the index of the
rows for the generation currently being processed, say generation $g$.  
We add correct computations to the matrix one row at a time;
all of the calculations depend only on the prior rows with the
exception of the [i,i] element.
This approach leads to
a for loop containing operations on single rows/columns.  

At one point below we use a vectorized version. It looks like the snippet below
\begin{nwchunk}
\nwhypn{notused}=
 for (g in 1:max(depth)) \{
     indx <- which(depth==g)
     kmat[indx,] <- (kmat[mother[indx],] + kmat[father[indx], ])/2
     kmat[,indx] <- (kmat[,mother[indx]] + kmat[,father[indx],])/2
     for (j in indx) kmat[j,j] <- (1 + kmat[mother[j], father[j]])/2
 \}
\end{nwchunk}
The first line computes all the values for a horizontal stripe of the
matrix. It will be correct for columns in generations $<g$, unreliable
for generation $g$ with itself because of incomplete parental relationships,
and zero for higher generations.
The second line does the vertical stripe, and because of the line before it
does have the data it needs and so gets all the stripe correct.
Except of course for the diagonal elements, for which formula \eqref{recur1}
does not hold.  We fill those in last.
We know that vectorized calculations are always faster in R and I was excited
to figure this out.  The unfortunate truth is that for this code
it hardly makes a difference, and for the X chromosome calculation leads to
impenetrable if-then-else logic.

The program can be called with a pedigree, a pedigree list, or
raw data.  The first argument is \Verb!id! instead of the more generic \Verb?x?
for backwards compatability with an older version of the routine.
We give founders a fake parent of subject $n+1$ who is not related to
anybody (even themself); it avoids some if-then-else constructions.
\begin{nwchunk}
\nwhypf{kinship1}{kinship}{kinship2}=
 kinship <- function(id, ...) \{
     UseMethod('kinship')
     \}
 
 kinship.default <- function(id, dadid, momid, sex, chrtype="autosome", ...) \{
     chrtype <- match.arg(casefold(chrtype), c("autosome", "x"))
     if (any(duplicated(id))) stop("All id values must be unique")
     n <- length(id)
     pdepth <- kindepth(id, dadid, momid)
     if (chrtype == "autosome") \{
         if (n==1) 
             return(matrix(.5,1,1, dimnames=list(id, id)))
 
         kmat <- diag(c(rep(.5, n), 0))  #founders
 
         mrow <- match(momid, id, nomatch=n+1) #row number of the mother
         drow <- match(dadid, id, nomatch=n+1) #row number of the dad 
         ## When all unrelateds, pdepth all=0. 
         ## Put c(1,) to make guard from iter 1:0
         for (depth in 1:max(c(1,pdepth))) \{
             for (j in  (1:n)[pdepth==depth]) \{
                 kmat[,j] <-kmat[j,] <- (kmat[mrow[j],]  + kmat[drow[j],]) /2
                 kmat[j,j] <- (1 + kmat[mrow[j], drow[j]]) /2
             \}
         \}
     \}
     else if (chrtype == "x") \{
         if (missing(sex) || length(sex) !=n) 
             stop("invalid sex vector")
         #1 = female, 2=male
         if (n==1) 
             return(matrix(ifelse(sex>2,sex/2,NA), 1,1, dimnames=list(id, id)))
 
         # kmat <- diag(c((3-sex)/2, 0)) #founders
         kmat <- diag(ifelse(sex>2, NA, c((3-sex)/2, 0)))
         mrow <- match(momid, id, nomatch=n+1) #row number of the mother
         drow <- match(dadid, id, nomatch=n+1) #row number of the dad 
 
         for (depth in 1:max(c(1,pdepth))) \{
             for (j in (1:n)[pdepth==depth]) \{
                 if (sex[j] ==1) \{
                     kmat[,j] <- kmat[j,] <- kmat[mrow[j],]
                     kmat[j,j]<- 1
                 \} 
                 else if(sex[j] == 2) \{
                     kmat[,j] <-kmat[j,] <- (kmat[mrow[j],]  + kmat[drow[j],]) /2
                     kmat[j,j] <- (1 + kmat[mrow[j], drow[j]]) /2
                 \} 
                 else \{
                     kmat[,j] <-kmat[j,] <- NA
                     kmat[j,j] <- NA 
                 \}
             \}
         \}
     \}
     kmat <- kmat[1:n,1:n]
     dimnames(kmat) <- list(id, id)
     kmat
 \}
\end{nwchunk}

The method for a pedigree object is an almost trivial modification.  Since the
mother and father are already indexed into the id list it has 
two lines that are different, those that create mrow and drow.
The other change is that now we potentially have information available
on monozygotic twins.  If there are any such, then when the second
twin of a pair is added to the matrix, we need to ensure that the
pair's kinship coefficient is set to the self-self value.
This can be done after each level is complete, but before children
for that level are computed.
If there are monozygotic triples, quadruplets, etc. this computation gets 
more involved.

The total number of monozygotic twins is always small, so it is efficient to
fix up all the monzygotic twins at each generation.
A variable \Verb!havemz! is set to TRUE if there are any, and an index array
\Verb!mzindex! is created for matrix subscripting.

\begin{nwchunk}
\nwhyp{kinship2}{kinship}{kinship1}{kinship3}=
 kinship.pedigree <- function(id, chrtype="autosome", ...) \{
     chrtype <- match.arg(casefold(chrtype), c("autosome", "x"))
     if (any(duplicated(id$id))) stop("All id values must be unique")
     n <- length(id$id)
     pdepth <- kindepth(id)
     
     # Are there any MZ twins to worry about?
     havemz <- FALSE
     if (!is.null(id$relation) && any(id$relation$code=="MZ twin")) \{
         havemz <- TRUE
         \nwhypf{makemzindex1}{makemzindex}{makemzindex2}
     \}
     
     if (chrtype == "autosome") \{
         if (n==1) 
             return(matrix(.5,1,1, dimnames=list(id$id, id$id)))
 
         kmat <- diag(c(rep(.5, n), 0))  #founders
         mrow <- ifelse(id$mindex ==0, n+1, id$mindex)
         drow <- ifelse(id$findex ==0, n+1, id$findex)
 
         for (depth in 1:max(pdepth)) \{
             indx <- which(pdepth == depth)
             kmat[indx,] <- (kmat[mrow[indx],] + kmat[drow[indx],]) /2
             kmat[,indx] <- (kmat[,mrow[indx]] + kmat[,drow[indx]]) /2
             for (j in indx) kmat[j,j] <- (1 + kmat[mrow[j], drow[j]])/2
             if (havemz) kmat[mzindex] <- (diag(kmat))[mzindex[,1]]
         \}
     \}
     else if (chrtype == "x") \{
         sex <- as.numeric(id$sex) # 1 = female, 2=male
         if (n==1) 
             return(matrix(sex/2, 1,1, dimnames=list(id$id, id$id)))
 
         kmat <- diag(c((3-sex)/2, 0))  #1 for males, 1/2 for females
         mrow <- ifelse(id$mindex ==0, n+1, id$mindex)
         drow <- ifelse(id$findex ==0, n+1, id$findex)
 
         for (depth in 1:max(pdepth)) \{
             for (j in (1:n)[pdepth==depth]) \{
                 if (sex[j] ==1) \{
                     kmat[,j] <- kmat[j,] <- kmat[mrow[j],]
                     kmat[j,j]<- 1
                 \}
                 else if(sex[j]==2) \{
                     kmat[,j] <-kmat[j,] <- (kmat[mrow[j],]  + kmat[drow[j],]) /2
                     kmat[j,j] <- (1 + kmat[drow[j],mrow[j]]) /2
                 \} else \{
                   kmat[,j] <-kmat[j,] <- NA
                    kmat[j,j] <- NA
                 \}
             if (havemz) kmat[mzindex] <- (diag(kmat))[mzindex[,1]]
             \}
         \}
     \}
     kmat <- kmat[1:n,1:n]
     dimnames(kmat) <- list(id$id, id$id)
     kmat
 \}
\end{nwchunk}

For the Minnesota Family Cancer Study there are 461 families and 29114
subjects.  The raw kinship matrix would be 29114 by 29114 which is over
5 terabytes of memory, something that clearly won't work within S.       %'
The solution is to store the overall matrix as a sparse Matrix object.
Each family forms a single block.  For this study we have
\Verb!n <- table(minnbreast$famid); sum(n*(n+1)/2)! or 1.07 million entries;
assuming that only the lower half of each matrix is stored.
The actual size is actually smaller than this, since each family's
matrix will have zeros in it --- founders for instance are not related ---
and those zeros are also not stored.

The result of each per-family call to kinship will be a symmetric matrix.
We first turn each of these into a dsCMatrix object, a sparse symmetric
form. 
The \Verb!bdiag! function is then used to paste all of these individual
sparse matrices into a single large matrix.

Why don't we use \Verb!(i in famlist)! below?  A numeric subscript of \Verb+[9]+ %'
selects the ninth family, not the family labeled as 9, so a numeric
family id would not act as we wished.
If all of the subject ids are unique, across all families, the final
matrix is labeled with the subject id, otherwise it is labeled with
family/subject.
\begin{nwchunk}
\nwhyp{kinship3}{kinship}{kinship2}{kinship4}=
 kinship.pedigreeList <- function(id, chrtype="autosome", ...) \{
     famlist <- unique(id$famid)
     nfam <- length(famlist)
     matlist <- vector("list", nfam)
     idlist  <- vector("list", nfam) #the possibly reorderd list of id values
    
     for (i in 1:length(famlist)) \{
         tped <- id[i]  #pedigree for this family
         temp <- try(kinship(tped, chrtype=chrtype, ...), silent=TRUE)
         if (class(temp)=="try-error") 
             stop(paste("In family", famlist[i], ":", temp))
         else matlist[[i]] <- as(forceSymmetric(temp), "dsCMatrix")
         idlist[[i]] <- tped$id
     \}
 
     result <- bdiag(matlist)
     if (any(duplicated(id$id))) 
         temp <-paste(rep(famlist, sapply(idlist, length)),
                      unlist(idlist), sep='/') 
     else temp <- unlist(idlist)
         
     dimnames(result) <- list(temp, temp)
     result
 \}
\end{nwchunk}

The older \Verb!makekinship! function,
from before the creation of pedigreeList objects,
accepts the raw identifier data, along with a special family code
for unrelated subjects, as produced by the \Verb!makefamid! function.
All the unrelated subjects are put at the front of the kinship matrix
in this case rather than within the family.
Because unrelateds get put into a fake family, we cannot create a
rational family/subject identifier; the id must be unique across
families.
We include a copy of the routine for backwards compatability, but
do not anticipate any new usage of it.
Like most routines, this starts out with a collection of error checks.
\begin{nwchunk}
\nwhypn{makekinship}=
 makekinship <- function(famid, id, father.id, mother.id, unrelated=0) \{
     n <- length(famid)
     if (length(id)    != n) stop("Mismatched lengths: famid and id")
     if (length(mother.id) != n) stop("Mismatched lengths: famid and mother.id")
     if (length(father.id) != n) stop("Mismatched lengths: famid and father.id")
     if (any(is.na(famid)))  stop("One or more subjects with missing family id")
     if (any(is.na(id)))     stop("One or more subjects with a missing id")
     if (is.numeric(famid)) \{
         if (any(famid <0))      stop("Invalid family id, must be >0")
         \}
 
     if (any(duplicated(id))) stop("Subject ids must be unique")
 
     famlist <- sort(unique(famid))  #same order as the counts table
     idlist <- id            # will be overwritten, but this makes it the
                             #  correct data type and length
     counts <- table(famid)
     cumcount <- cumsum(counts)    
      if (any(famid==unrelated)) \{
         # Assume that those with famid of 0 are unrelated uniques
         #   (usually the marry-ins)
         temp <- match(unrelated, names(counts))
         nzero <- counts[temp]    
         counts <- counts[-temp]
         famlist <- famlist[famlist != unrelated]
         idlist[1:nzero] <- id[famid== unrelated]
         cumcount <- cumsum(counts) + nzero
         \}
     else nzero <- 0
     
     mlist <- vector('list', length(counts))
     for (i in 1:length(counts)) \{
         who <- (famid == famlist[i])
         if (sum(who) ==1) mlist[[i]] <- Matrix(0.5)  # family of size 1
         else \{
             mlist[[i]] <- kinship(id[who], mother.id[who], father.id[who])
             \}
         idlist[seq(to=cumcount[i], length=counts[i])] <- id[who]
         \}
 
     if (nzero>0) mlist <- c(list(Diagonal(nzero)), mlist)
     kmat <- forceSymmetric(bdiag(mlist))
     dimnames(kmat) <- list(idlist, idlist)
     kmat
 \}
\end{nwchunk}

Return now to the question of monzygotic sets.
Consider the following rather difficult example:
\begin{verbatim}
   1  2
   1  3
   5  6
   3  7
   10 9
\end{verbatim}
Subjects 1, 2, 3, and 7 form a monozygotic quadruple, 5/6 and 9/10 are
monzygotic pairs.  
First create a vector \code{mzgrp} which contains for each subject the
lowest index of a monozygotic twin for that subject.  
For non-twins it can have any value.  
For this example that vector is set to 1 for subjects 1, 2, 3, and 7,
to 5 for 5 and 6, and to 9 for 9 and 10.
Creating this requires a short while loop.
Once this is in hand we can identify the sets.
\begin{nwchunk}
\nwhyp{makemzindex2}{makemzindex}{makemzindex1}{makemzindex3}=
 temp <- which(id$relation$code=="MZ twin")
 ## drop=FALSE added in case only one MZ twin set
 mzmat <- as.matrix(id$relation[,c("indx1", "indx2")])[temp,,drop=FALSE]
 mzgrp <- 1:max(mzmat) #everyone starts in their own group
 # The loop below will take k-1 iterations for a set labeled as
 #   (k-1):k, ..., 4:3, 3:2, 2:1;  this is the worst case.
 while(1) \{
     if (all(mzgrp[mzmat[,1]] == mzgrp[mzmat[,2]])) break
     for (i in 1:nrow(mzmat)) 
         mzgrp[mzmat[i,1]] <- mzgrp[mzmat[i,2]] <- min(mzgrp[mzmat[i,]])
     \}
\end{nwchunk}
Now make a matrix that has a row for every possible pair.
Finally, remove the rows that are identical.
The result is a set of all pairs of observations in the matrix that
correspond to monozygotic pairs.
\begin{nwchunk}
\nwhypb{makemzindex3}{makemzindex}{makemzindex2}=
 mzindex <- cbind(unlist(tapply(mzmat, mzgrp[mzmat], function(x) \{
                                 z <- unique(x)
                                 rep(z, length(z))\})),
                  unlist(tapply(mzmat, mzgrp[mzmat], function(x) \{
                                 z <- unique(x)
                                 rep(z, each=length(z))\})))
 mzindex <- mzindex[mzindex[,1] != mzindex[,2],]
\end{nwchunk}

\section{Older routines}
For testing purposes we have a version of the kinship function prior to
addition of the chrtype argument.


\begin{nwchunk}
\nwhypb{kinship4}{kinship}{kinship3}=
 oldkinship <- function(id, ...) \{
     UseMethod('oldkinship')
     \}
 
 oldkinship.default <- function(id, dadid, momid, ...) \{
     n <- length(id)
     if (n==1) 
         return(matrix(.5,1,1, dimnames=list(id, id)))
     if (any(duplicated(id))) stop("All id values must be unique")
     kmat <- diag(n+1) /2
     kmat[n+1,n+1]    <- 0 
 
     pdepth <- kindepth(id, dadid, momid)
     mrow <- match(momid, id, nomatch=n+1) #row number of the mother
     drow <- match(dadid, id, nomatch=n+1) #row number of the dad 
 
     for (depth in 1:max(pdepth)) \{
         indx <- (1:n)[pdepth==depth]
         for (i in indx) \{
             mom <- mrow[i]
             dad <- drow[i]
             kmat[i,]  <- kmat[,i] <- (kmat[mom,] + kmat[dad,])/2
             kmat[i,i] <- (1+ kmat[mom,dad])/2
             \}
         \}
     
     kmat <- kmat[1:n,1:n]
     dimnames(kmat) <- list(id, id)
     kmat
     \}
 
 oldkinship.pedigree <- function(id, ...) \{
     n <- length(id$id)
     if (n==1) 
         return(matrix(.5,1,1, dimnames=list(id$id, id$id)))
     if (any(duplicated(id$id))) stop("All id values must be unique")
     kmat <- diag(n+1) /2
     kmat[n+1,n+1]    <- 0 
 
     pdepth <- kindepth(id)
     mrow <- ifelse(id$mindex ==0, n+1, id$mindex)
     drow <- ifelse(id$findex ==0, n+1, id$findex)
 
     # Are there any MZ twins to worry about?
     if (!is.null(id$relation) && any(id$relation$code=="MZ twin")) \{
         havemz <- TRUE
         temp <- which(id$relation$code=="MZ twin")
         ## drop=FALSE added in case only one MZ twin set
         mzmat <- as.matrix(id$relation[,c("indx1", "indx2")])[temp,,drop=FALSE]
 
         # any triples, quads, etc?
         if (any(table(mzmat) > 1)) \{ #yes there are
             # each group id will be min(member id)
             mzgrp <- 1:max(mzmat)  #each person a group
             indx <- sort(unique(as.vector(mzmat)))
             # The loop below will take k-1 iterations for a set labeled as
             #   1:2, 2:3, ...(k-1):k;  this is the worst case.
             while(1) \{
                 z1 <- mzgrp[mzmat[,1]]
                 z2 <- mzgrp[mzmat[,2]]
                 if (all(z1 == z2)) break
                 mzgrp[indx] <- tapply(c(z1, z1, z2, z2), c(mzmat,mzmat), min)
             \}
             # Now mzgrp = min person id for each person in a set
             matlist <- tapply(mzmat, mzgrp[mzmat], function(x) \{
                 x <- sort(unique(x))
                 temp <- cbind(rep(x, each=length(x)), rep(x, length(x)))
                 temp[temp[,1] != temp[,2],]
                 \})
             \}
         else \{  #no triples, easier case
             matlist <- tapply(mzmat, row(mzmat), function(x) 
                             matrix(x[c(1,2,2,1)],2), simplify=FALSE)
             \}
         \}
     else havemz <- FALSE
 
     for (depth in 1:max(pdepth)) \{
         indx <- (1:n)[pdepth==depth]
         for (i in indx) \{
             mom <- mrow[i]
             dad <- drow[i]
             kmat[i,]  <- kmat[,i] <- (kmat[mom,] + kmat[dad,])/2
             kmat[i,i] <- (1+ kmat[mom,dad])/2
             \}
         if (havemz) \{
             for (i in 1:length(matlist)) \{
                 temp <- matlist[[i]]
                 kmat[temp] <- kmat[temp[1], temp[1]]
             \}
         \}
     \}
     
     kmat <- kmat[1:n,1:n]
     dimnames(kmat) <- list(id$id, id$id)
     kmat
 \}    
 
 oldkinship.pedigreeList <- function(id, ...) \{
     famlist <- unique(id$famid)
     nfam <- length(famlist)
     matlist <- vector("list", nfam)
     idlist  <- vector("list", nfam) #the possibly reorderd list of id values
    
     for (i in 1:length(famlist)) \{
         tped <- id[i]  #pedigree for this family
         temp <- try(oldkinship(tped, ...), silent=TRUE)
         if (class(temp)=="try-error") 
             stop(paste("In family", famlist[i], ":", temp))
         else matlist[[i]] <- as(forceSymmetric(temp), "dsCMatrix")
         idlist[[i]] <- tped$id
     \}
 
     result <- bdiag(matlist)
     if (any(duplicated(id$id))) 
         temp <-paste(rep(famlist, sapply(idlist, length)),
                      unlist(idlist), sep='/') 
     else temp <- unlist(idlist)
         
     dimnames(result) <- list(temp, temp)
     result
 \}
\end{nwchunk}
\section{Pedigree alignment}
An \emph{aligned} pedigree is an object that contains a pedigree along
with a set of information that allows for pretty plotting.
This information consists of two parts: 
a set of vertical and horizontal plotting coordinates along with the
identifier of the subject to be plotted at each position,
and a list of connections to be made between parent/child, spouse/spouse,
and twin/twin.
Creating this aligment turned out to be one of the more difficult parts
of the project, and is the area where significant further work could be
done.  
All the routines in this section completely ignore the \Verb!id! component
of a pedigree; everyone is indexed solely by their row number in the object.

\subsection{Hints}

The first part of the work has to do with a \Verb!hints! list for each
pedigree.  It consists of 3 parts:
\begin{itemize}
  \item The left to right order in which founders should be processed.
  \item The order in which siblings should be listed within a family.
  \item For selected spouse pairs, who is on the left/right, and which of the
    two should be the anchor, i.e., determine where the marriage is plotted.
    \end{itemize}
The default starting values for all of these are simple: founders are 
processed in the order in which they appear in the data set, 
children appear in the order they are found in the data set,
husbands are to the left of their wives, and a marriage is plotted
at the leftmost spouse.
A simple example where we want to bend these rules is when two families
marry, and the pedigrees for both extend above the wedded pair.  
In the joint pedigree the
pair should appear as the right-most child in the left hand family, and
as the left-most child in the right hand family.
With respect to founders, assume that a family has three lineages with
a marriage between 1 and 2, and another between 2 and 3.  In the joint
pedigree the sets should be 1, 2, 3 from left to right.  

The hints consist of a list with two components.
The first is a vector of numbers of the same length as the pedigree,
used to order the female founders and to order siblings within
family.  For subjects not part of either of these the value can be 
arbitrary.  
The second is a 3 column matrix of spouse pairs, each row indicates the
left-hand member of the pair, the right-hand member, and which of the two
is the anchor, i.e., directly connected to thier parent.
Double and triple marriages can start to get interesting.


The \Verb!autohint! routine is used to create an initial hints list.
It is a part of the general intention to make the routine do
``pretty good'' drawings automatically.                 
The basic algorithm is trial and error. 
\begin{itemize}
  \item Start with the simplest possible hints (user input is accepted)
  \item Call align.pedigree to see how this works out
  \item Fix any spouses that are not next to each other but could be.
  \item Any fix on the top level mixes up everything below, so we do the
    fixes one level at a time.
\end{itemize}
The routine makes no attempt to reorder founders.  It just isn't smart enough%'
to figure that out.

The first thing to be done is to check on twins.  They are a nuisance, since
twins need to move together.  The \Verb!ped$relation! object has a factor in it, 
so first turn that into numeric.
We create 3 vectors: \Verb!twinrel! is a matrix containing pairs of twins and
their relation, it is a subset of the incoming \Verb!relation! matrix.
The \Verb!twinset! vector identifies twins, it is 0 for anyone who is not a 
part of a multiple-birth set, and a unique id for each member of a set.  
We use the minimum row number of the members of the set as the id.
\Verb!twinord! is a starting order vector for the set; it mostly makes sure
that there are no ties (who knows what a user may have used for starting 
values.)  

A recent addition (JPS, 5/2013) is to carry forward packaged and align to 
kindepth and align.pedigree.

\begin{nwchunk}
\nwhypn{autohint}=
 autohint <- function(ped, hints, packed=TRUE, align=FALSE) \{
     if (!is.null(ped$hints)) return(ped$hints)  #nothing to do
     n <- length(ped$id)
     depth <- kindepth(ped, align=TRUE)
 
     if (is.null(ped$relation)) relation <- NULL
     else  relation <- cbind(as.matrix(ped$relation[,1:2]), 
                             as.numeric(ped$relation[,3]))
     if (!is.null(relation) && any(relation[,3] <4)) \{
         temp <- (relation[,3] < 4)
         twinlist <- unique(c(relation[temp,1:2]))  #list of twin id's 
         twinrel  <- relation[temp,,drop=F]
         
         twinset <- rep(0,n)
         twinord <- rep(1,n)
         for (i in 2:length(twinlist)) \{
             # Now, for any pair of twins on a line of twinrel, give both
             #  of them the minimum of the two ids
             # For a set of triplets, it might take two iterations for the
             #  smallest of the 3 numbers to "march" across the threesome.
             #  For quads, up to 3 iterations, for quints, up to 4, ....
             newid <- pmin(twinrel[,1], twinrel[,2])
             twinset[twinrel[,1]] <- newid
             twinset[twinrel[,2]] <- newid
             twinord[twinrel[,2]] <- pmax(twinord[twinrel[,2]], 
                                          twinord[twinrel[,1]]+1)
             \}        
         \}
     else \{
         twinset <- rep(0,n)
         twinrel <- NULL
         \}
     \nwhypf{autohint-shift1}{autohint-shift}{autohint-shift2}
     \nwhypf{autohint-init1}{autohint-init}{autohint-init2}
     \nwhypf{autohint-fixup1}{autohint-fixup}{autohint-fixup2}
     list(order=horder, spouse=sptemp)    
     \}
\end{nwchunk}

Next is an internal function that  rearranges someone to be
the leftmost or rightmost of his/her siblings.  The only
real complication is twins -- if one of them moves the other has to move too.  
And we need to keep the monozygotics together within a band of triplets.
Algorithm: if the person to be moved is part of a twinset, 
first move all the twins to the left end (or right
as the case may be), then move all the monozygotes to the
left, then move the subject himself to the left.
\begin{nwchunk}
\nwhypb{autohint-shift2}{autohint-shift}{autohint-shift1}=
 shift <- function(id, sibs, goleft, hint, twinrel, twinset) \{
     if (twinset[id]> 0)  \{ 
         shift.amt <- 1 + diff(range(hint[sibs]))  # enough to avoid overlap
         twins <- sibs[twinset[sibs]==twinset[id]]
         if (goleft) 
              hint[twins] <- hint[twins] - shift.amt
         else hint[twins] <- hint[twins] + shift.amt
                 
         mono  <- any(twinrel[c(match(id, twinrel[,1], nomatch=0),
                                match(id, twinrel[,2], nomatch=0)),3]==1)
         if (mono) \{
             #
             # ok, we have to worry about keeping the monozygotics
             #  together within the set of twins.
             # first, decide who they are, by finding those monozygotic
             #  with me, then those monozygotic with the results of that
             #  iteration, then ....  If I were the leftmost, this could
             #  take (#twins -1) iterations to get us all
             #
             monoset <- id
             rel2 <- twinrel[twinrel[,3]==1, 1:2, drop=F]
             for (i in 2:length(twins)) \{
                 newid1 <- rel2[match(monoset, rel2[,1], nomatch=0),2]
                 newid2 <- rel2[match(monoset, rel2[,2], nomatch=0),1]
                 monoset <- unique(c(monoset, newid1, newid2))
                 \}
             if (goleft) 
                    hint[monoset]<- hint[monoset] - shift.amt
             else   hint[monoset]<- hint[monoset] + shift.amt
             \}
         \}
 
     #finally, move the subject himself
     if (goleft) hint[id] <- min(hint[sibs]) -1   
     else        hint[id] <- max(hint[sibs]) +1
 
     hint[sibs] <- rank(hint[sibs])  # aesthetics -- no negative hints
     hint
     \}
\end{nwchunk}

Now, get an ordering of the pedigree to use as the starting point.  
The numbers start at 1 on each level.
We don't need the final ``prettify" step, hence align=F.
If there is a hints structure entered, we retain it's non-zero entries,
otherwise people are put into the order of the data set. 
We allow the hints input to be only an order vector
Twins are
then further reordered.
\begin{nwchunk}
\nwhypb{autohint-init2}{autohint-init}{autohint-init1}=
 if (!missing(hints)) \{
     if (is.vector(hints)) hints <- list(order=hints)
     if (is.matrix(hints)) hints <- list(spouse=hints)
     if (is.null(hints$order)) horder <- integer(n)
     else horder <- hints$order
     \}
 else horder <- integer(n)
 
 for (i in unique(depth)) \{
     who <- (depth==i & horder==0)  
     if (any(who)) horder[who] <- 1:sum(who) #screwy input - overwrite it
     \}
 
 if (any(twinset>0)) \{
     # First, make any set of twins a cluster: 6.01, 6.02, ...
     #  By using fractions, I don't have to worry about other sib's values
     for (i in unique(twinset)) \{
         if (i==0) next
         who <- (twinset==i)
         horder[who] <- mean(horder[who]) + twinord[who]/100
         \}
 
     # Then reset to integers
     for (i in unique(ped$depth)) \{
         who <- (ped$depth==i)
         horder[who] <- rank(horder[who])  #there should be no ties
         \}
     \}
 
 if (!missing(hints)) sptemp <- hints$spouse
 else sptemp <- NULL
 plist <- align.pedigree(ped, packed=packed, align=align, 
                         hints=list(order=horder, spouse=sptemp))
\end{nwchunk}
The result coming back from align.pedigree is a set of vectors and
matrices:
\begin{description}
  \item[n] vector, number of entries per level
  \item[nid] matrix, one row per level, numeric id of the subject plotted
    here
  \item[spouse] integer matrix, one row per level, subject directly to my
    right is my spouse (1), a double marriage (2), or neither (0).
  \item[fam] matrix, link upward to my parents, or 0 if no link.
\end{description}

\begin{figure}
  \myfig{autohint1}
  \caption{A simple pedigree before (left) and after (right) the
    autohint computations.}
  \label{fig:auto1}
\end{figure}

Now, walk down through the levels one by one.
A candidate subject is one who appears twice on the level, once
under his/her parents and once somewhere else as a spouse.
Move this person and spouse the the ends of their sibships and
add a marriage hint.
Figure \ref{fig:auto1} shows a simple case.  The input data set has
the subjects ordered from 1--11, the left panel is the result without
hints which processes subjects in the order encountered.
The return values from \Verb!align.pedigree! have subject 9 shown twice.
The first is when he is recognized as the spouse of subject 4, the second
as the child of 6--7.

The basic logic is
\begin{enumerate}
  \item Find a subject listed multiple times on a line (assume it is a male).
    This means that he has multiple connections, usually one to his parents and
    the other to a spouse tied to her parents.  (If the
    spouse were a marry-in she would have been placed alongside and there
    would be no duplication.)
  \item Say subject x is listed at locations 2, 8, and 12.  We look at one
    pairing at a time, either 2-8 or 8-12.  Consider the first one.
    \begin{itemize}
      \item If position 2 is associated with siblings, rearrange them to
        put subject 2 on the right.  If it is associated with a spouse at
        this location, put that spouse on the right of her siblings.
      \item Repeat the work for position 8, but moving targets to the left.
      \item At either position, if it is associated with a spouse then
        add a marriage.  If both ends of the marriage are anchored, i.e.,
        connected to a family, then either end may be listed as the anchor
        in the output; follow the suggestion of the duporder routine.  If
        only one is, it is usually better to anchor it there, so that the
        marriage is processed by\Verb!align.pedigree! when that family is.
        (At least I think so.)
    \end{itemize}
\end{enumerate}
This logic works 9 times out of 10, at least for human pedigrees.
We'll look at more complex cases below when looking at the \Verb!duporder!   %'
(order the duplicates)
function, which returns a matrix with columns 1 and 2 being a pair
of duplicates, and 3 a direction.
Note that in the following code \Verb!idlist! refers to the row numbers of
each subject in the pedigree, not to their label \Verb!ped$id!.
\begin{nwchunk}
\nwhyp{autohint-fixup2}{autohint-fixup}{autohint-fixup1}{autohint-fixup3}=
 \nwhypf{autohint-find1}{autohint-find}{autohint-find2}
 \nwhypf{autohint-duporder1}{autohint-duporder}{autohint-duporder2}
 maxlev <- nrow(plist$nid)
 for (lev in 1:maxlev) \{
     idlist <- plist$nid[lev,1:plist$n[lev]] #subjects on this level
     dpairs <- duporder(idlist, plist, lev, ped)  #duplicates to be dealt with
     if (nrow(dpairs)==0) next;  
     for (i in 1:nrow(dpairs)) \{
         anchor <- spouse <- rep(0,2)
         for (j in 1:2) \{
             direction <- c(FALSE, TRUE)[j]
             mypos <- dpairs[i,j]
             if (plist$fam[lev, mypos] >0) \{
                 # Am connected to parents at this location
                 anchor[j] <- 1  #familial anchor
                 sibs <- idlist[findsibs(mypos, plist, lev)]
                 if (length(sibs) >1) 
                     horder <- shift(idlist[mypos], sibs, direction, 
                                     horder, twinrel, twinset)
                 \}
             else \{
                 #spouse at this location connected to parents ?
                 spouse[j] <- findspouse(mypos, plist, lev, ped)
                 if (plist$fam[lev,spouse[j]] >0) \{ # Yes they are
                     anchor[j] <- 2  #spousal anchor
                     sibs <- idlist[findsibs(spouse[j], plist, lev)]
                     if (length(sibs) > 1) 
                         horder <- shift(idlist[spouse[j]], sibs, direction, 
                                     horder, twinrel, twinset)
                     \}
                 \}
             \}
\end{nwchunk}

At this point the most common situation will be what is shown in 
figure \ref{fig:auto1}.  The variable \Verb!anchor! is (2,1) showing that the
left hand copy of subject 9 is connected to an anchored spouse and the
right hand copy is himself anchored.  The proper addition to the
spouselist is \Verb!(4, 9, dpairs)!, where the last is the hint from the
dpairs routine as to which of the parents is the one to follow further when
drawing the entire pedigree.  (When drawing a pedigree and there is a
child who can be reached from multiple founders, we only want to find
the child once.) 

The double marry-in found in figure \ref{fig:auto2}, subject 11, leads
to value of (2,2) for the \Verb!anchor! variable.  The proper addition to
the \Verb!sptemp! matrix in this case will be two rows, (5, 11, 1) indicating
that 5 should be plotted left of 11 for the 5-11 marriage, with the first
partner as the anchor, and a second row (11, 9, 2).
This will cause the common spouse to be plotted in the middle.

Multiple marriages can lead to unanchored subjects.  
In the left hand portion of figure \ref{fig:auto3} we have two
double marriages, one on the left and one on the right with 
anchor values of (0,2) and (2,0), respectively.  
We add two marriages to the return list to ensure that both print
in the correct left-right order; the 14-4 one is correct by default
but it's easier to output a line than check sex orders.  %'

\begin{figure}
  \myfig{autohint3}
  \caption{Pedigrees with multiple marriages}
  \label{fig:auto3}
  \end{figure}

The left panel of figure \ref{fig:auto3} shows a case where
subject 11 marries into the pedigree but also has a second spouse.
The \Verb!anchor! variable for
this case will be (2, 0); the first instance of 11 has a spouse tied
into the tree above, the second instance has no upward connections.
In the top row, subject 6 has values of (0, 0) since neither 
connection has an upward parent.  
In the right hand panel subject 2 has an anchor variable of (0,1).

\begin{nwchunk}
\nwhypb{autohint-fixup3}{autohint-fixup}{autohint-fixup2}=
         # add the marriage(s)
         id1 <- idlist[dpairs[i,1]]  # i,1 and i,2 point to the same person
         id2 <- idlist[spouse[1]]
         id3 <- idlist[spouse[2]]
 
         temp <- switch(paste(anchor, collapse=''),
                        "21" = c(id2, id1, dpairs[i,3]),   #the most common case
                        "22" = rbind(c(id2, id1, 1), c(id1, id3, 2)),
                        "02" = c(id2, id1, 0), 
                        "20" = c(id2, id1, 0), 
                        "00" = rbind(c(id1, id3, 0), c(id2, id1, 0)),
                        "01" = c(id2, id1, 2),
                        "10" = c(id1, id2, 1),
                        NULL)
 
         if (is.null(temp)) \{ 
             warning("Unexpected result in autohint, please contact developer")
             return(list(order=1:n))  #punt
           \}         
         else sptemp <- rbind(sptemp, temp)
         \}
     #
     # Recompute, since this shifts things on levels below
     #
     plist <- align.pedigree(ped, packed=packed, align=align, 
                             hints=list(order=horder, spouse=sptemp))   
     \}
\end{nwchunk}

For the case shown in figure \ref{fig:align1} the \Verb!duporder! function
will return a single row array with values (2, 6, 1), the first two
being the positions of the duplicated subject.  
The anchor will be 2 since that is the copy connected to parents
The direction is TRUE, since the spouse is to the left of the anchor point.
The id is 9, sibs are 8, 9, 10, and the shift function will create position
hints of 2,1,3, which will cause them to be listed in the order 9, 8, 10.

The value of spouse is 3 (third position in the row), subjects 3,4, and 5
are reordered, and finally the line (4,9,1) is added to the sptemp 
matrix.  
In this particular case the final element could be a 1 or a 2, since both
are connected to their parents.

\begin{figure}
  \myfig{autohint2}
  \caption{A more complex pedigree.}
  \label{fig:align2}
\end{figure}

Figure \ref{fig:align2} shows a more complex case with several arcs.
In the upper left is a double marry-in.
The \Verb!anchor! variable in the above code
will be (2,2) since both copies have an anchored spouse.
The left and right sets of sibs are reordered (even though the left
one does not need it), and two lines are added to the sptemp matrix:
(5,11,1) and (11,9,2).

On the upper right is a pair of overlapping arcs.
In the final tree we want to put sibling 28 to the right of 29 since
that will allow one node to join, but if we process the subjects in
lexical order the code will first shift 28 to the right and then later
shift over 29.
The duporder function tries to order the duplicates into a matrix
so that the closest ones are processed last.  The definition of close
is based first on whether the families touch, and second on the
actual distance.
The third column of the matrix hints at whether the marriage should
be plotted at the left (1) or right (2) position of the pair.  The
goal for this is to spread apart families of cousins; in the
example to not have the children of 28/31 plotted under the 21/22
grandparents, and those for 29/32 under the 25/26 grandparents. 
The logic for this column is very ad hoc: put children near the edges.
\begin{nwchunk}
\nwhypb{autohint-duporder2}{autohint-duporder}{autohint-duporder1}=
 duporder <- function(idlist, plist, lev, ped) \{
     temp <- table(idlist)
     if (all(temp==1)) return (matrix(0L, nrow=0, ncol=3))
     
     # make an intial list of all pairs's positions
     # if someone appears 4 times they get 3 rows
     npair <- sum(temp-1)
     dmat <- matrix(0L, nrow=npair, ncol=3)
     dmat[,3] <- 2; dmat[1:(npair/2),3] <- 1
     i <- 0
     for (id in unique(idlist[duplicated(idlist)])) \{
         j <- which(idlist==id)
         for (k in 2:length(j)) \{
             i <- i+1
             dmat[i,1:2] <- j[k + -1:0]
             \}
         \}
     if (nrow(dmat)==1) return(dmat)  #no need to sort it
     
     # families touch?
     famtouch <- logical(npair)
     for (i in 1:npair) \{
         if (plist$fam[lev,dmat[i,1]] >0) 
              sib1 <- max(findsibs(dmat[i,1], plist, lev))
         else \{
             spouse <- findspouse(dmat[i,1], plist, lev, ped)
             ##If spouse is marry-in then move on without looking for sibs
                 if (plist$fam[lev,spouse]==0) \{famtouch[i] <- F; next\}
             sib1 <- max(findsibs(spouse, plist, lev)) 
             \}
         
         if (plist$fam[lev, dmat[i,2]] >0)
             sib2 <- min(findsibs(dmat[i,2], plist, lev))
         else \{
             spouse <- findspouse(dmat[i,2], plist, lev, ped)
             ##If spouse is marry-in then move on without looking for sibs
                 if (plist$fam[lev,spouse]==0) \{famtouch[i] <- F; next\}
             sib2 <- min(findsibs(spouse, plist, lev))
             \}
         famtouch[i] <- (sib2-sib1 ==1)
         \}
     dmat[order(famtouch, dmat[,1]- dmat[,2]),, drop=FALSE ]
     \}
\end{nwchunk}

Finally, here are two helper routines.
Finding my spouse can be interesting -- suppose we have a listing with
Shirley, Fred, Carl, me on the line with the first three marked as
spouse=TRUE -- it means that she has been married to all 3 of us.
First we find the string from rpos to lpos that is a marriage block;
99\% of the time this will be of length 2 of course.  Then find
the person in that block who is opposite sex, and check that they
are connected.
The routine is called with a left-right position in the alignment
arrays and returns a position.
\begin{nwchunk}
\nwhyp{autohint-find2}{autohint-find}{autohint-find1}{autohint-find3}=
 findspouse <- function(mypos, plist, lev, ped) \{
     lpos <- mypos
     while (lpos >1 && plist$spouse[lev, lpos-1]) lpos <- lpos-1
     rpos <- mypos
     while(plist$spouse[lev, rpos]) rpos <- rpos +1
     if (rpos==lpos) stop("autohint bug 3")
     
     opposite <-ped$sex[plist$nid[lev,lpos:rpos]] != ped$sex[plist$nid[lev,mypos]]
     if (!any(opposite)) stop("autohint bug 4")  # no spouse
     spouse <- min((lpos:rpos)[opposite])  #can happen with a triple marriage
     spouse
     \}
\end{nwchunk}

The findsibs function starts with a position and returns a position as well.
\begin{nwchunk}
\nwhypb{autohint-find3}{autohint-find}{autohint-find2}=
 findsibs <- function(mypos, plist, lev) \{
     family <- plist$fam[lev, mypos]
     if (family==0) stop("autohint bug 6")
     which(plist$fam[lev,] == family)
     \}
\end{nwchunk}


\subsection{Align.pedigree}
\label{sect:alignped}
The top level routine for alignment has 5 arguments
\begin{description}
    \item[ped] a pedigree or pedigreeList object. In the case of
      the latter we loop over each family separately.
    \item[packed] do we allow branches of the tree to overlap?  
      If FALSE the drawing is much easier, but final drawing can
      take up a huge amount of space.  
    \item[width] the minimum width for a packed pedigree. This
      affects only small pedigrees, since the minimum possible
      width for a pedigree is the largest number of individiuals in
      one of the generations.
    \item[align] should the final step of alignment be done?  This
      tries to center children under parents, to the degree possible.
    \item a hints object.  This is normally blank and autohint
      is invoked. 
\end{description}
The result coming back from align.pedigree is a set of vectors and
matrices:
\begin{description}
  \item[n] vector, number of entries per level
  \item[nid] matrix, one row per level, numeric id of the subject plotted
    here
  \item[pos] the horizontal position for plotting
  \item[spouse] integer matrix, one row per level, subject directly to my
    right is my spouse (1), a double marriage (2), or neither (0).
  \item[fam] matrix, link upward to my parents, or 0 if no link.
\end{description}
\begin{nwchunk}
\nwhypn{align.pedigree}=
 align.pedigree <- function(ped, packed=TRUE, width=10,
                            align=TRUE, hints=ped$hints) \{
     if (class(ped)== 'pedigreeList') \{
         nped <- length(unique(ped$famid))
         alignment <- vector('list', nped)
         for (i in 1:nped) \{
             temp <- align.pedigree(ped[i], packed, width, align)
             alignment[[i]] <- temp$alignment
             \}
         ped$alignment <- alignment
         class(ped) <- 'pedigreeListAligned'
         return(ped)
         \}
     
     if (is.null(hints)) \{
       hints <- try(\{autohint(ped)\}, silent=TRUE)
       ## sometimes appears dim(ped) is empty (ped is NULL), so try fix here: (JPS 6/6/17
       if(class(hints)=="try-error") hints <- list(order=seq_len(max(1, dim(ped)))) ## 1:dim(ped))
     \} else \{
       hints <- check.hint(hints, ped$sex)
     \}
     
     \nwhypf{align-setup1}{align-setup}{align-setup2}
     \nwhypf{align-founders1}{align-founders}{align-founders2}
     \nwhypf{align-finish1}{align-finish}{align-finish2}
     \}
\end{nwchunk}


Start with some setup.  
Throughout this routine the row number is used as a subject
id (ignoring the actual id label).
\begin{itemize}
  \item Check that everyone has either two
    parents or none (a singleton will just confuse us).
  \item Verify that the hints are correct.
  \item The relation data frame, if present, has a factor in it.  Turn
    that into numeric.
\item Create the \Verb!spouselist! array.  This has 4 columns
  \begin{enumerate}
    \item Husband index (4= 4th person in the pedigree structure)
    \item Wife index
    \item Plot order: 1= husband left, 2=wife left
    \item Anchor: 1=left member, 2=right member, 0= not yet determined
      \end{enumerate}
  As the routine proceeds a spousal pair can be encountered
  multiple times; we take them out of this list when the ``connected''
  member is added to the pedigree so that no marriage gets added
  twice.  
\item To detect duplicates on the spouselist we need to create a
  unique (but temporary) spouse-pair id using a simple hash.
\end{itemize}

When importing data from autohint, that routine's spouse matrix %'
has column 1 =
subject plotted on the left, 2 = subject plotted on the right.
The \Verb!spouselist! array has column 1=husband, 2=wife.  
Hence the clumsy looking ifelse below.  The autohint format is more
congenial to users, who might modify the output, the spouselist format
easier for the code.


\begin{nwchunk}
\nwhypb{align-setup2}{align-setup}{align-setup1}=
 n <- length(ped$id)
 dad <- ped$findex; mom <- ped$mindex  #save typing
 if (any(dad==0 & mom>0) || any(dad>0 & mom==0))
         stop("Everyone must have 0 parents or 2 parents, not just one")
 level <- 1 + kindepth(ped, align=TRUE)
 
 horder <- hints$order   # relative order of siblings within a family
 
 if (is.null(ped$relation)) relation <- NULL
 else  relation <- cbind(as.matrix(ped$relation[,1:2]), 
                         as.numeric(ped$relation[,3]))
 
 if (!is.null(hints$spouse)) \{ # start with the hints list
     tsex <- ped$sex[hints$spouse[,1]]  #sex of the left member
     spouselist <- cbind(0,0,  1+ (tsex!='male'), 
                         hints$spouse[,3])
     spouselist[,1] <- ifelse(tsex=='male', hints$spouse[,1], hints$spouse[,2])
     spouselist[,2] <- ifelse(tsex=='male', hints$spouse[,2], hints$spouse[,1])
     \}
 else spouselist <- matrix(0L, nrow=0, ncol=4)
 
 if (!is.null(relation) && any(relation[,3]==4)) \{
     # Add spouses from the relationship matrix
     trel <- relation[relation[,3]==4,,drop=F]
     tsex <- ped$sex[trel[,1]]
     trel[tsex!='male',1:2] <- trel[tsex!='male',2:1]
     spouselist <- rbind(spouselist, cbind(trel[,1],
                                           trel[,2],
                                           0,0))
     \}
 if (any(dad>0 & mom>0) ) \{
     # add parents
     who <- which(dad>0 & mom>0)
     spouselist <- rbind(spouselist, cbind(dad[who], mom[who], 0, 0))
     \}
 
 hash <- spouselist[,1]*n + spouselist[,2]
 spouselist <- spouselist[!duplicated(hash),, drop=F]
\end{nwchunk}

The \Verb!alignped! routine does the alignment using 3 co-routines:
\begin{description}
  \item[alignped1] called with a single subject, returns the subtree
    founded on this subject, as though it were the only tree
  \item[alignped2] called with a set of sibs, calls alignped1 and 
    alignped3 multiple times to create a joint pedigree
  \item[alignped3] given two side by side plotting structures, merge them
    into a single one
\end{description}
 
Call \Verb!alignped1! sequentially with each founder pair and merge the
results.  
A founder pair is a married pair, neither of which has a father.

\begin{nwchunk}
\nwhypb{align-founders2}{align-founders}{align-founders1}=
 noparents <- (dad[spouselist[,1]]==0 & dad[spouselist[,2]]==0)
  ##Take duplicated mothers and fathers, then founder mothers
 dupmom <- spouselist[noparents,2][duplicated(spouselist[noparents,2])] #Founding mothers with multiple marriages
 dupdad <- spouselist[noparents,1][duplicated(spouselist[noparents,1])] #Founding fathers with multiple marriages
 foundmom <- spouselist[noparents&!(spouselist[,1] %in% c(dupmom,dupdad)),2] # founding mothers
 founders <-  unique(c(dupmom, dupdad, foundmom))    
 founders <-  founders[order(horder[founders])]  #use the hints to order them
 rval <- alignped1(founders[1], dad, mom, level, horder, 
                           packed=packed, spouselist=spouselist)
 
 if (length(founders)>1) \{
     spouselist <- rval$spouselist
     for (i in 2:length(founders)) \{
         rval2 <- alignped1(founders[i], dad, mom,
                            level, horder, packed, spouselist)
         spouselist <- rval2$spouselist
         rval <- alignped3(rval, rval2, packed)
         \}
     \}
\end{nwchunk}

Now finish up.  
There are 4 tasks to doS
\begin{enumerate}
  \item For convenience the lower level routines kept the spouse
    and nid arrays as a single object -- unpack them
  \item In the spouse array a 1 in position i indicates that subject
    i and i+1 are joined as a marriage.  If these two have a common
    ancestor change this to a 2, which indicates that a double line
    should be used in the plot.
  \item Add twins data to the output.
  \item Do final alignment
\end{enumerate}

\begin{nwchunk}
\nwhyp{align-finish2}{align-finish}{align-finish1}{align-finish3}=
 #
 # Unhash out the spouse and nid arrays
 #
 nid    <- matrix(as.integer(floor(rval$nid)), nrow=nrow(rval$nid))
 spouse <- 1L*(rval$nid != nid)
 maxdepth <- nrow(nid)
 
 # For each spouse pair, find out if it should be connected with
 #  a double line.  This is the case if they have a common ancestor
 ancestor <- function(me, momid, dadid) \{
     alist <- me
     repeat \{
         newlist <- c(alist, momid[alist], dadid[alist])
         newlist <- sort(unique(newlist[newlist>0]))
         if (length(newlist)==length(alist)) break
         alist <- newlist
         \}
     alist[alist!=me]
     \}
 for (i in (1:length(spouse))[spouse>0]) \{
     a1 <- ancestor(nid[i], mom, dad)
     a2 <- ancestor(nid[i+maxdepth],mom, dad)  #matrices are in column order
     if (any(duplicated(c(a1, a2)))) spouse[i] <- 2
     \}
\end{nwchunk}

The twins array is of the same shape as the spouse and nid arrays:
one row per level giving data for the subjects plotted on that row.
In this case they are
\begin{itemize}
  \item 0= nothing
  \item 1= the sib to my right is a monzygotic twin, 
  \item 2= the sib to my right is a dizygote,
  \item 3= the sib to my right is a twin, unknown zyogosity.
\end{itemize}
\begin{nwchunk}
\nwhyp{align-finish3}{align-finish}{align-finish2}{align-finish4}=
 if (!is.null(relation) && any(relation[,3] < 4)) \{
     twins <- 0* nid
     who  <- (relation[,3] <4)
     ltwin <- relation[who,1]
     rtwin <- relation[who,2]
     ttype <- relation[who,3]
     
     # find where each of them is plotted (any twin only appears
     #   once with a family id, i.e., under their parents)
     ntemp <- ifelse(rval$fam>0, nid,0) # matix of connected-to-parent ids
     ltemp <- (1:length(ntemp))[match(ltwin, ntemp, nomatch=0)]
     rtemp <- (1:length(ntemp))[match(rtwin, ntemp, nomatch=0)]
     twins[pmin(ltemp, rtemp)] <- ttype
     \}
 else twins <- NULL
\end{nwchunk}
 
At this point the pedigree has been arranged, with the positions
in each row going from 1 to (number of subjects in the row).
(For a packed pedigree, which is the usual case).
Having everything pushed to the left margin isn't very
pretty, now we fix that.
Note that alignped4 wants a T/F spouse matrix: it doesn't care
  about your degree of relationship to the spouse.
\begin{nwchunk}
\nwhypb{align-finish4}{align-finish}{align-finish3}=
 if ((is.numeric(align) || align) && max(level) >1) 
     pos <- alignped4(rval, spouse>0, level, width, align)
 else pos <- rval$pos
 
 if (is.null(twins))
      list(n=rval$n, nid=nid, pos=pos, fam=rval$fam, spouse=spouse)
 else list(n=rval$n, nid=nid, pos=pos, fam=rval$fam, spouse=spouse, 
               twins=twins)
\end{nwchunk}
\subsection{alignped1}
This is the first of the three co-routines.
It is called with a single subject, and returns the subtree founded
on said subject, as though it were the only tree.  
We only go down the pedigree, not up.
Input arguments are
\begin{description}
  \item[nid] the numeric id of the subject in question
  \item[dad] points to the row of the father, 0=no father in pedigree
  \item[mom] points to the row of the mother
  \item[level] the plotting depth of each subject
  \item[horder] orders the kids within a sibship
  \item[packed] if true, everything is slid to the left
  \item[spouselist] a matrix of spouses
    \begin{itemize}
      \item col 1= pedigree index of the husband
      \item col 2= pedigree index of the wife
      \item col 3= 1:plot husband to the left, 2= wife to the left
      \item col 4= 1:left member is rooted here, 2=right member, 0=either
    \end{itemize}
\end{description}

The return argument is a set of matrices as described in 
section \ref{sect:alignped}, along with the spouselist matrix.
The latter has marriages removed as they are processed.

In this routine the \Verb!nid! array consists of the final nid array + 1/2 of the
final spouse array.
The basic algorithm is simple.  
\begin{enumerate}
  \item Find all of the spouses for which \Verb!x! is the anchor subject.  If
    there are none then return the trivial tree consisting of \Verb!x! alone.
  \item For each marriage in the set, call \Verb!alignped2! on the children
    and add this to the result.
\end{enumerate}
Note that the \Verb!spouselist! matrix will only contain spouse pairs that
are not yet processed.
The logic for anchoring is slightly tricky.  First, if row 4 of
the spouselist matrix is 0, we anchor at the first opportunity, i.e. now..
Also note that if spouselist[,3]==spouselist[,4] it is
the husband who is the anchor (just write out the possibilities).

\begin{nwchunk}
\nwhypf{alignped11}{alignped1}{alignped12}=
 alignped1 <- function(x, dad, mom, level, horder, packed, spouselist)\{
     # Set a few constants
     maxlev <- max(level)
     lev <- level[x]
     n <- integer(maxlev)
 
     if (length(spouselist)==0)  spouse <- NULL
     else \{
         if (any(spouselist[,1]==x))\{
             sex <- 1                                  # I'm male
             sprows <- (spouselist[,1]==x & (spouselist[,4] ==spouselist[,3] |
                                             spouselist[,4] ==0))
             spouse <- spouselist[sprows, 2] #ids of the spouses
             \}
         else \{
             sex <- 2
             sprows <- (spouselist[,2]==x & (spouselist[,4]!=spouselist[,3] |
                                             spouselist[,4] ==0))
             spouse <- spouselist[sprows, 1]
             \}
         \}
     # Marriages that cross levels are plotted at the higher level (lower
     #  on the paper).
     if (length(spouse)) \{
         keep <- level[spouse] <= lev
         spouse <- spouse[keep]
         sprows <- (which(sprows))[keep]
         \}
     nspouse <- length(spouse)  # Almost always 0, 1 or 2
\end{nwchunk}
Create the set of 3 return structures, which will be matrices with
(1+nspouse) columns.
If there are children then other routines will widen the result.
\begin{nwchunk}
\nwhyp{alignped12}{alignped1}{alignped11}{alignped13}=
     nid <- fam <- matrix(0L, maxlev, nspouse+1)
     pos <- matrix(0.0, maxlev, nspouse +1)
     n[lev] <- nspouse +1       
     pos[lev,] <- 0:nspouse
     if (nspouse ==0) \{   
         # Easy case: the "tree rooted at x" is only x itself
         nid[lev,1] <- x
         return(list(nid=nid, pos=pos, fam=fam, n=n, spouselist=spouselist))
         \}
\end{nwchunk}
Now we have a list of spouses that should be dealt with and 
the the correponding columns of the spouselist matrix.  
Create the two complimentary lists lspouse and rspouse to denote
those plotted on the left and on the right.  
For someone with lots of spouses we try to split them evenly.
If the number of spouses is odd, then men should have more on the
right than on the left, women more on the right.
Any hints in the spouselist matrix override.
We put the undecided marriages closest to \Verb!x!, 
then add predetermined ones to the left and
right.
The majority of marriages will be undetermined singletons, for which
nleft will be 1 for female (put my husband to the left) and 0 for male.
In one bug found by plotting canine data, lspouse could initially be empty but 
length(rspouse)> 1. This caused nleft>length(indx). A fix was to not let
indx to be indexed beyond its length, fix by JPS 5/2013.

\begin{nwchunk}
\nwhyp{alignped13}{alignped1}{alignped12}{alignped14}=
     lspouse <- spouse[spouselist[sprows,3] == 3-sex] # 1-2 or 2-1
     rspouse <- spouse[spouselist[sprows,3] == sex]   # 1-1 or 2-2
     if (any(spouselist[sprows,3] ==0)) \{
         #Not yet decided spouses
         indx <- which(spouselist[sprows,3] ==0)
         nleft <- floor((length(sprows) + (sex==2))/2) #total number to left
         nleft <- nleft - length(lspouse)  #number of undecideds to the left
         if (nleft >0) \{
             # JPS fixed 5/2013, don't index when nleft > length(indx)
             lspouse <- c(lspouse, spouse[indx[seq_len(min(nleft,length(indx)))]])
             indx <- indx[-(seq_len(min(nleft,length(indx))))]       
           \}
         if (length(indx)) rspouse <- c(spouse[indx], rspouse)
       \}
 
     nid[lev,] <- c(lspouse, x, rspouse)
     nid[lev, 1:nspouse] <- nid[lev, 1:nspouse] + .5  #marriages    
 
     spouselist <- spouselist[-sprows,, drop=FALSE]
\end{nwchunk}

The spouses are in the pedigree, now look below.
For each spouse get the list of children.
If there are any we call alignped2 to generate their tree and
then mark the connection to their parent.
If multiple marriages have children we need to join the
trees.
\begin{nwchunk}
\nwhyp{alignped14}{alignped1}{alignped13}{alignped15}=
     nokids <- TRUE   #haven't found any kids yet
     spouse <- c(lspouse, rspouse)  #reorder
     for (i in 1:nspouse) \{
         ispouse <- spouse[i]
         children <- which((dad==x & mom==ispouse) | (dad==ispouse & mom==x))
         if (length(children) > 0) \{
             rval1 <- alignped2(children, dad, mom, level, horder, 
                               packed, spouselist)
             spouselist <- rval1$spouselist
             # set the parentage for any kids
             #  a nuisance: it's possible to have a child appear twice, when
             #  via inbreeding two children marry --- makes the "indx" line
             #  below more complicated
             temp <- floor(rval1$nid[lev+1,])  # cut off the .5's for matching
             indx <- (1:length(temp))[match(temp,children, nomatch=0) >0]
             rval1$fam[lev+1,indx] <- i   #set the kids parentage
             if (!packed) \{
                 # line the kids up below the parents
                 # The advantage at this point: we know that there is 
                 #   nothing to the right that has to be cared for
                 kidmean <- mean(rval1$pos[lev+1, indx])
                 parmean <- mean(pos[lev, i + 0:1])
                 if (kidmean > parmean) \{
                     # kids to the right of parents: move the parents
                     indx <- i:(nspouse+1)
                     pos[lev, indx] <- pos[lev, indx] + (kidmean - parmean)
                     \}
                 else \{
                     # move the kids and their spouses and all below
                     shift <- parmean - kidmean
                     for (j in (lev+1):maxlev) \{
                         jn <- rval1$n[j]
                         if (jn>0) 
                             rval1$pos[j, 1:jn] <- rval1$pos[j, 1:jn] +shift
                         \}
                     \}
                 \}
             if (nokids) \{
                 rval <- rval1
                 nokids <- FALSE
                 \}
             else \{
                 rval <- alignped3(rval, rval1, packed)
                 \}
             \}
         \}
\end{nwchunk}

To finish up we need to splice together the tree made up
from all the kids, which only has data from lev+1 down,
with the data here.  
There are 3 cases.  The first and easiest is when no
children were found.
The second, and most common, is when the tree below is
wider than the tree here, in which case we add the
data from this level onto theirs.
The third is when below is narrower, for instance an
only child.
\begin{nwchunk}
\nwhypb{alignped15}{alignped1}{alignped14}=
     if (nokids) \{
         return(list(nid=nid, pos=pos, fam=fam, n=n, spouselist=spouselist))
         \}
 
     if (ncol(rval$nid) >= 1+nspouse) \{
         # The rval list has room for me!
         rval$n[lev] <- n[lev]
         indx <- 1:(nspouse+1)
         rval$nid[lev, indx] <- nid[lev,]
         rval$pos[lev, indx] <- pos[lev,]
         \}
     else \{
         #my structure has room for them
         indx <- 1:ncol(rval$nid)   
         rows <- (lev+1):maxlev
         n[rows] <- rval$n[rows]
         nid[rows,indx] <- rval$nid[rows,]
         pos[rows,indx] <- rval$pos[rows,]
         fam[rows,indx] <- rval$fam[rows,]
         rval <- list(nid=nid, pos=pos, fam=fam, n=n)
         \}
     rval$spouselist <- spouselist
     rval
     \}
\end{nwchunk}

\subsection{alignped2}
This routine takes a collection of siblings, grows the tree for
each, and appends them side by side into a single tree.
The input arguments are the same as those to
\Verb!alignped1! with the exception that \Verb?x? will be a vector.
This routine does nothing to the spouselist matrix, but needs
to pass it down the tree and back since one of the routines
called by \Verb!alignped2! might change the matrix.

The code below has one non-obvious special case.  Suppose
that two sibs marry.  
When the first sib is processed by \Verb!alignped1! then both
partners (and any children) will be added to the rval
structure below.  
When the second sib is processed they
will come back as a 1 element tree (the marriage will no longer
be on the spouselist), which should \emph{not} be added
onto rval.  
The rule thus is to not add any 1 element tree whose
value (which must be x[i]) is already in the rval structure for this level.
(Where did Curtis O. \emph{find} these families?)

\begin{nwchunk}
\nwhypn{alignped2}=
 alignped2 <- function(x, dad, mom, level, horder, packed,
                       spouselist) \{
     x <- x[order(horder[x])]  # Use the hints to order the sibs
     rval <- alignped1(x[1],  dad, mom, level, horder, packed, 
                       spouselist)
     spouselist <- rval$spouselist
 
     if (length(x) >1) \{
         mylev <- level[x[1]]
         for (i in 2:length(x)) \{
             rval2 <-  alignped1(x[i], dad, mom, level,
                                 horder, packed, spouselist)
             spouselist <- rval2$spouselist
             
             # Deal with the unusual special case:
             if ((rval2$n[mylev] > 1) || 
                           (is.na(match(x[i], floor(rval$nid[mylev,])))))
                 rval <- alignped3(rval, rval2, packed)
             \}
         rval$spouselist <- spouselist
         \}
     rval
     \}
\end{nwchunk}


\subsection{alignped3}
The third co-routine merges two pedigree trees which are side by
side into a single object.
The primary special case is when the rightmost person in the left
tree is the same as the leftmost person in the right tree; we 
needn't plot two copies of the same person side by side.
(When initializing the output structures don't worry about this - there
is no harm if they are a column bigger than finally needed.)
Beyond that the work is simple bookkeeping.

\begin{nwchunk}
\nwhypn{alignped3}=
 alignped3 <- function(x1, x2, packed, space=1) \{
     maxcol <- max(x1$n + x2$n)
     maxlev <- length(x1$n)
     n1 <- max(x1$n)   # These are always >1
     n  <- x1$n + x2$n
 
     nid <- matrix(0, maxlev, maxcol)
     nid[,1:n1] <- x1$nid
     
     pos <- matrix(0.0, maxlev, maxcol)
     pos[,1:n1] <- x1$pos
 
     fam <- matrix(0, maxlev, maxcol)
     fam[,1:n1] <- x1$fam
     fam2 <- x2$fam
     if (!packed) \{
         \nwhypf{align3-slide1}{align3-slide}{align3-slide2}
         \}
     \nwhypf{align3-merge1}{align3-merge}{align3-merge2}
 
     if (max(n) < maxcol) \{
         maxcol <- max(n)
         nid <- nid[,1:maxcol]
         pos <- pos[,1:maxcol]
         fam <- fam[,1:maxcol]
         \}
 
     list(n=n, nid=nid, pos=pos, fam=fam)
     \}
\end{nwchunk}

For the unpacked case, which is the traditional way to draw a pedigree
when we can assume the paper is infinitely wide, all parents are centered
over their children.  
In this case we think if the two trees to be merged as solid blocks.
On input they both have a left margin of 0.
Compute how far over we have to slide the right tree.
\begin{nwchunk}
\nwhypb{align3-slide2}{align3-slide}{align3-slide1}=
 slide <- 0
 for (i in 1:maxlev) \{
     n1 <- x1$n[i]
     n2 <- x2$n[i]
     if (n1 >0 & n2 >0) \{
         if (nid[i,n1] == x2$nid[i,1])
                 temp <- pos[i, n1] - x2$pos[i,1]
         else    temp <- space + pos[i, n1] - x2$pos[i,1]
         if (temp > slide) slide <- temp
         \}
     \}
\end{nwchunk}

Now merge the two trees. 
Start at the top level and work down.
\begin{enumerate}
  \item If n2=0, there is nothing to do
  \item Decide if there is a subject overlap, and if so 
    \begin{itemize}
      \item Set the proper parent id. 
        Only one of the two copies will be attached and the other
        will have fam=0, so max(fam, fam2) preserves the correct one.
      \item If not packed, set the position.  Choose the one connected
        to a parent, or midway for a double marriage.
    \end{itemize}
  \item If packed=TRUE determine the amount of slide for this row. It
    will be \Verb!space! over from the last element in the left pedigree,
    less overlap.
  \item Move everything over
  \item Fix all the children of this level, right hand pedigree, to
    point to the correct parental position.
\end{enumerate}

\begin{nwchunk}
\nwhypb{align3-merge2}{align3-merge}{align3-merge1}=
 for (i in 1:maxlev) \{
     n1 <- x1$n[i]
     n2 <- x2$n[i]
     if (n2 >0) \{   # If anything needs to be done for this row...
         if (n1>0 && (nid[i,n1] == floor(x2$nid[i,1]))) \{
             #two subjects overlap
             overlap <- 1
             fam[i,n1] <- max(fam[i,n1], fam2[i,1])
             nid[i,n1] <- max(nid[i,n1], x2$nid[i,1]) #preserve a ".5"
             if (!packed) \{
                 if(fam2[i,1]>0) 
                     if (fam[i,n1]>0) 
                         pos[i,n1] <- (x2$pos[i,1] + pos[i,n1] + slide)/2
                     else pos[i,n1] <- x2$pos[i,1]+ slide
                     \}
             n[i] <- n[i] -1
             \}
         else overlap <- 0
         
         if (packed) slide <- if (n1==0) 0 else pos[i,n1] + space - overlap
 
         zz <- seq(from=overlap+1, length=n2-overlap)
         nid[i, n1 + zz- overlap] <- x2$nid[i, zz]
         fam[i, n1 + zz -overlap] <- fam2[i,zz] 
         pos[i, n1 + zz -overlap] <- x2$pos[i,zz] + slide
         
         if (i<maxlev) \{
                 # adjust the pointers of any children (look ahead)
             temp <- fam2[i+1,]
             fam2[i+1,] <- ifelse(temp==0, 0, temp + n1 -overlap)
                 \}
         \}
     \}
\end{nwchunk}

\section{alignped4}
The alignped4 routine is the final step of alignment.  It attempts to line
up children under parents and put spouses and siblings `close' to each other,%'` 
to the extent possible within the constraints of page width.  This routine
used to be the most intricate and complex of the set, until I realized that
the task could be cast as constrained quadradic optimization.
The current code does necessary setup and then calls the \Verb!quadprog!
function.  
At one point I investigated using one of the simpler least-squares routines
where $\beta$ is constrained to be non-negative. 
However a problem can only be translated into that form if the number
of constraints is less than the number of parameters, which is not
true in this problem.

There are two important parameters for the function.  One is the user specified
maximum width.  The smallest possible width is the maximum number of subjects
on a line, if the user's suggestion  %'
is too low it is increased to that 1+ that
amount (to give just a little wiggle room).
The other is a vector of 2 alignment parameters $a$ and $b$.
For each set of siblings ${x}$ with parents at $p_1$ and $p_2$ the
alignment penalty is
$$
   (1/k^a)\sum{i=1}{k} (x_i - (p_1 + p_2)^2
$$
where $k$ is the number of siblings in the set.
Using the fact that $\sum(x_i-c)^2 = \sum(x_i-\mu)^2 + k(c-\mu)^2$,
when $a=1$ then moving a sibship with $k$ sibs one unit to the left or
right of optimal will incur the same cost as moving one with only 1 or
two sibs out of place.  If $a=0$ then large sibships are harder to move
than small ones, with the default value $a=1.5$ they are slightly easier 
to move than small ones.  The rationale for the default is as long as the
parents are somewhere between the first and last siblings the result looks
fairly good, so we are more flexible with the spacing of a large family.
By tethering all the sibs to a single spot they tend are kept close to 
each other.
The alignment penalty for spouses is $b(x_1 - x_2)^2$, which tends to keep 
them together.  The size of $b$ controls the relative importance of sib-parent
and spouse-spouse closeness.

We start by adding in these penalties.  The total number of parameters
in the alignment problem (what we hand to quadprog) is the set 
of \Verb!sum(n)! positions.  A work array myid keeps track of the parameter
number for each position so that it is easy to find.
There is one extra penalty added at the end.  Because the penalty amount
would be the same if all the final positions were shifted by a constant,
the penalty matrix will not be positive definite; solve.QP doesn't like  %'
this.  We add a tiny amount of leftward pull to the widest line.
\begin{nwchunk}
\nwhypf{alignped41}{alignped4}{alignped42}=
 alignped4 <- function(rval, spouse, level, width, align) \{
     if (is.logical(align)) align <- c(1.5, 2)  #defaults
     maxlev <- nrow(rval$nid)
     width <- max(width, rval$n+.01)   # width must be > the longest row
 
     n <- sum(rval$n)  # total number of subjects
     myid <- matrix(0, maxlev, ncol(rval$nid))  #number the plotting points
     for (i in 1:maxlev) \{
         myid[i, rval$nid[i,]>0] <-  cumsum(c(0, rval$n))[i] + 1:rval$n[i]
         \}
 
     # There will be one penalty for each spouse and one for each child
     npenal <- sum(spouse[rval$nid>0]) + sum(rval$fam >0) 
     pmat <- matrix(0., nrow=npenal+1, ncol=n)
 
     indx <- 0
     # Penalties to keep spouses close
     for (lev in 1:maxlev) \{
         if (any(spouse[lev,])) \{
             who <- which(spouse[lev,])
             indx <- max(indx) + 1:length(who)
             pmat[cbind(indx, myid[lev,who])] <-  sqrt(align[2])
             pmat[cbind(indx, myid[lev,who+1])] <- -sqrt(align[2])
             \}
         \}
 
     # Penalties to keep kids close to parents
     for (lev in (1:maxlev)[-1])  \{ # no parents at the top level
         families <- unique(rval$fam[lev,])
         families <- families[families !=0]  #0 is the 'no parent' marker
         for (i in families) \{  #might be none
             who <- which(rval$fam[lev,] == i)
             k <- length(who)
             indx <- max(indx) +1:k   #one penalty per child
             penalty <- sqrt(k^(-align[1]))
             pmat[cbind(indx, myid[lev,who])] <- -penalty
             pmat[cbind(indx, myid[lev-1, rval$fam[lev,who]])] <- penalty/2
             pmat[cbind(indx, myid[lev-1, rval$fam[lev,who]+1])] <- penalty/2
             \}
         \}
     maxrow <- min(which(rval$n==max(rval$n)))
     pmat[nrow(pmat), myid[maxrow,1]] <- 1e-5
\end{nwchunk}

Next come the constraints.  If there are $k$ subjects on a line there will
be $k+1$ constraints for that line.  The first point must be $\ge 0$, each
subesquent one must be at least 1 unit to the right, and the final point
must be $\le$ the max width.
\begin{nwchunk}
\nwhypb{alignped42}{alignped4}{alignped41}=
     ncon <- n + maxlev    # number of constraints
     cmat <- matrix(0., nrow=ncon, ncol=n)
     coff <- 0  # cumulative constraint lines so var
     dvec <- rep(1., ncon)
     for (lev in 1:maxlev) \{
         nn <- rval$n[lev]
         if (nn>1) \{
             for (i in 1:(nn-1)) 
                 cmat[coff +i, myid[lev,i + 0:1]] <- c(-1,1)
             \}
 
         cmat[coff+nn,   myid[lev,1]]  <- 1     #first element >=0
         dvec[coff+nn] <- 0
         cmat[coff+nn+1, myid[lev,nn]] <- -1    #last element <= width-1
         dvec[coff+nn+1] <- 1-width
         coff <- coff + nn+ 1
         \}
 
     if (exists('solve.QP')) \{
          pp <- t(pmat) %*% pmat + 1e-8 * diag(ncol(pmat))
          fit <- solve.QP(pp, rep(0., n), t(cmat), dvec)
          \}
     else stop("Need the quadprog package")
 
     newpos <- rval$pos
     #fit <- lsei(pmat, rep(0, nrow(pmat)), G=cmat, H=dvec)
     #newpos[myid>0] <- fit$X[myid]           
     newpos[myid>0] <- fit$solution[myid]
     newpos
     \}
\end{nwchunk}
\section{Plots}
The plotting function for pedigrees has 5 tasks
\begin{enumerate}
  \item Gather information and check the data.  
    An important step is the call to align.pedigree.
  \item Set up the plot region and size the symbols.  
    The program wants to plot circles and squares,
    so needs to understand the geometry of the paper, pedigree size, and text 
    size to get the right shape and size symbols.
  \item Set up the plot and add the symbols for each subject
  \item Add connecting lines between spouses, and children with parents
  \item Create an invisible return value containing the locations.
\end{enumerate}
Another task, not yet completely understood, is how we might break a plot 
across multiple pages.

\begin{nwchunk}
\nwhypn{plot.pedigree}=
 plot.pedigree <- function(x, id = x$id, status = x$status, 
                           affected = x$affected, 
                           cex = 1, col = 1,
                           symbolsize = 1, branch = 0.6, 
                           packed = TRUE, align = c(1.5,2), width = 8, 
                           density=c(-1, 35,65,20), mar=c(4.1, 1, 4.1, 1),
                           angle=c(90,65,40,0), keep.par=FALSE,
                           subregion, pconnect=.5, ...)
 \{
     Call <- match.call()
     n <- length(x$id)        
     \nwhypf{pedplot-data1}{pedplot-data}{pedplot-data2}
     \nwhypf{pedplot-sizing1}{pedplot-sizing}{pedplot-sizing2}
     \nwhypf{pedplot-symbols1}{pedplot-symbols}{pedplot-symbols2}
     \nwhypf{pedplot-lines1}{pedplot-lines}{pedplot-lines2}
     \nwhypf{pedplot-finish1}{pedplot-finish}{pedplot-finish2}
     \}
\end{nwchunk}

\subsection{Setup}
The dull part is first: check all of the input data for
correctness.  
The \Verb!sex! variable is taken from the pedigree so we need not check
that. 
The identifier for each subject is by default the \Verb!id! variable from
the pedigree, but users often want to add some extra text.
The status variable can be used to put a line through the symbol
of those who are deceased, it is an optional part of the pedigree.
\begin{nwchunk}
\nwhyp{pedplot-data2}{pedplot-data}{pedplot-data1}{pedplot-data3}=
 if(is.null(status))
   status <- rep(0, n)
 else \{
     if(!all(status == 0 | status == 1))
       stop("Invalid status code")
     if(length(status) != n)
       stop("Wrong length for status")
 \}
 if(!missing(id)) \{
     if(length(id) != n)
       stop("Wrong length for id")
 \}
\end{nwchunk}
The ``affected status'' is a 0/1 matrix of any marker data that the
user might want to add.  It may be attached to the pedigree or added
here.  It can be a vector of length \Verb!n! or a matrix with \Verb?n? rows.
If it is not present, the default is to print open symbols without
shading or color, which corresponds to a code of 0, while a 1 means to
shade the symbol.  

If the argment is a matrix, then the shading and/or density value for
ith column is taken from the ith element of the angle/density arguments.

(Update by JPS 5/2011) Update to allow missing values (NA) in the ``affected''
indicators.  Missingness of affection status will have a ``?'' in 
the midpoint of the portion of the plot symbol rather than blank or shaded.
The ``?'' is in line with standards discussed in 
Bennet et a. J of Gent Couns., 2008.

For purposes within the plot method, NA values in ``affected'' are coded 
to -1.

\begin{nwchunk}
\nwhypb{pedplot-data3}{pedplot-data}{pedplot-data2}=
 if(is.null(affected))\{
   affected <- matrix(0,nrow=n)
 \}
 else \{
     if (is.matrix(affected))\{
         if (nrow(affected) != n) stop("Wrong number of rows in affected")
         if (is.logical(affected)) affected <- 1* affected
         if (ncol(affected) > length(angle) || ncol(affected) > length(density))
             stop("More columns in the affected matrix than angle/density values")
         \} 
     else \{
         if (length(affected) != n)
             stop("Wrong length for affected")
 
         if (is.logical(affected)) affected <- as.numeric(affected)
         if (is.factor(affected))  affected <- as.numeric(affected) -1
         \}
     if(max(affected, na.rm=TRUE) > min(affected, na.rm=TRUE)) \{
       affected <- matrix(affected - min(affected, na.rm=TRUE),nrow=n)
      ## affected[is.na(affected)] <- -1
     \} else \{
       affected <- matrix(affected,nrow=n)
     \}
     ## JPS 4/28/17 bug fix b/c some cases NAs are not set to -1
     affected[is.na(affected)] <- -1
     if (!all(affected==0 | affected==1 | affected == -1))
             stop("Invalid code for affected status")
 \}
 
 if (length(col) ==1) col <- rep(col, n)
 else if (length(col) != n) stop("Col argument must be of length 1 or n")
\end{nwchunk}

\subsection{Sizing}
Now we need to set the sizes. 
From align.pedigree we will get the maximum width and depth. 
There is one plotted row for each row of the returned matrices.
The number of columns of the matrices is the max width of the pedigree,
so there are unused positions in shorter rows, these can be identifed
by having an nid value of 0.
Horizontal locations for each point go from 0 to xmax, subjects are at
least 1 unit apart; a large number will be exactly one unit part.
These locations will be at the top center of each plotted symbol.
\begin{nwchunk}
\nwhyp{pedplot-sizing2}{pedplot-sizing}{pedplot-sizing1}{pedplot-sizing3}=
 \nwhypf{pedplot-subregion1}{pedplot-subregion}{pedplot-subregion2}
 plist <- align.pedigree(x, packed = packed, width = width, align = align)
 if (!missing(subregion)) plist <- subregion2(plist, subregion)
 xrange <- range(plist$pos[plist$nid >0])
 maxlev <- nrow(plist$pos)
\end{nwchunk}

We would like to to make the boxes about 2.5 characters wide, which matches
most labels, but no more than 0.9 units wide or .5 units high.  
We also want to vertical room for the labels. Which should have at least
1/2 of stemp2 space above and stemp2 space below.  
The stemp3 variable is the height of labels: users may use multi-line ones.
Our constraints then are
\begin{itemize}
  \item (box height + label height)*maxlev $\le$ height: the boxes and labels have
    to fit vertically
  \item (box height) * (maxlev + (maxlev-1)/2) $\le$ height: at least 1/2 a box of
    space between each row of boxes
  \item (box width) $\le$ stemp1 in inches 
  \item (box width) $\le$ 0.8 unit in user coordinates, otherwise they appear 
    to touch
  \item User coordinates go from min(xrange)- 1/2 box width to 
    max(xrange) + 1/2 box width.
  \item the box is square (in inches)
\end{itemize}

The first 3 of these are easy.  The fourth comes into play only for very packed
pedigrees. Assume that the box were the maximum size of .8 units, i.e., minimal
spacing between them. Then xmin -.45 to xmax + .45 covers the plot region,
the scaling between user coordinates and inches is (.8 + xmax-xmin) user =
figure region inches, and the box is .8*(figure width)/(.8 + xmax-xmin).
The transformation from user units to inches horizontally depends on the box
size, since I need to allow for 1/2 a box on the left and right.  
Vertically the range from 1 to nrow spans the tops of the symbols, which 
will be the figure region height less (the height of the
text for the last row + 1 box); remember that the coordinates point to the
top center of the box.
We want row 1 to plot at the top, which is done by appropriate setting
of the usr parameter.
\begin{nwchunk}
\nwhypb{pedplot-sizing3}{pedplot-sizing}{pedplot-sizing2}=
 frame()
 oldpar <- par(mar=mar, xpd=TRUE)
 psize <- par('pin')  # plot region in inches
 stemp1 <- strwidth("ABC", units='inches', cex=cex)* 2.5/3
 stemp2 <- strheight('1g', units='inches', cex=cex)
 stemp3 <- max(strheight(id, units='inches', cex=cex))
 
 ht1 <- psize[2]/maxlev - (stemp3 + 1.5*stemp2)
 if (ht1 <=0) stop("Labels leave no room for the graph, reduce cex")
 ht2 <- psize[2]/(maxlev + (maxlev-1)/2)
 wd2 <- .8*psize[1]/(.8 + diff(xrange))
 
 boxsize <- symbolsize* min(ht1, ht2, stemp1, wd2) # box size in inches
 hscale <- (psize[1]- boxsize)/diff(xrange)  #horizontal scale from user-> inch
 vscale <- (psize[2]-(stemp3 + stemp2/2 + boxsize))/ max(1, maxlev-1)
 boxw  <- boxsize/hscale  # box width in user units
 boxh  <- boxsize/vscale   # box height in user units
 labh  <- stemp2/vscale   # height of a text string
 legh  <- min(1/4, boxh  *1.5)  # how tall are the 'legs' up from a child
 par(usr=c(xrange[1]- boxw/2, xrange[2]+ boxw/2, 
           maxlev+ boxh+ stemp3 + stemp2/2 , 1))
\end{nwchunk}

\subsection{Drawing the tree}
Now we draw and label the boxes.  Definition of the drawbox function is
deferred until later.
\begin{nwchunk}
\nwhypb{pedplot-symbols2}{pedplot-symbols}{pedplot-symbols1}=
 \nwhypf{pedplot-drawbox1}{pedplot-drawbox}{pedplot-drawbox2}
 
 sex <- as.numeric(x$sex)
 for (i in 1:maxlev) \{
     for (j in seq_len(plist$n[i])) \{
         k <- plist$nid[i,j]
         drawbox(plist$pos[i,j], i, sex[k], affected[k,],
                 status[k], col[k], polylist, density, angle,
                 boxw, boxh)
         text(plist$pos[i,j], i + boxh + labh*.7, id[k], cex=cex, 
            adj=c(.5,1), ...)
         \}
 \}
\end{nwchunk}

Now draw in the connections one by one. First those between spouses.
\begin{nwchunk}
\nwhyp{pedplot-lines2}{pedplot-lines}{pedplot-lines1}{pedplot-lines3}=
 maxcol <- ncol(plist$nid)  #all have the same size
 for(i in 1:maxlev) \{
     tempy <- i + boxh/2
     if(any(plist$spouse[i,  ]>0)) \{
         temp <- (1:maxcol)[plist$spouse[i,  ]>0]
         segments(plist$pos[i, temp] + boxw/2, rep(tempy, length(temp)), 
                  plist$pos[i, temp + 1] - boxw/2, rep(tempy, length(temp)))
 
         temp <- (1:maxcol)[plist$spouse[i,  ] ==2]
         if (length(temp)) \{ #double line for double marriage
             tempy <- tempy + boxh/10
             segments(plist$pos[i, temp] + boxw/2, rep(tempy, length(temp)), 
                    plist$pos[i, temp + 1] - boxw/2, rep(tempy, length(temp)))
             \}
     \}
 \}
\end{nwchunk}
Now connect the children to the parents.  First there are lines up from each
child, which would be trivial except for twins, triplets, etc.  Then we 
draw the horizontal bar across siblings and finally the connector from
the parent.  For twins, the ``vertical'' lines are angled towards a 
common point, the variable is called \Verb!target! below.
The horizontal part is easier if we do things family by
family.  The \Verb!plist$twins! variable is 1/2/3 for a twin on my right,
0 otherwise.

\begin{nwchunk}
\nwhyp{pedplot-lines3}{pedplot-lines}{pedplot-lines2}{pedplot-lines4}=
 for(i in 2:maxlev) \{
     zed <- unique(plist$fam[i,  ])
     zed <- zed[zed > 0]  #list of family ids
     
     for(fam in zed) \{
         xx <- plist$pos[i - 1, fam + 0:1]
         parentx <- mean(xx)   #midpoint of parents
 
 
         # Draw the uplines
         who <- (plist$fam[i,] == fam) #The kids of interest
         if (is.null(plist$twins)) target <- plist$pos[i,who]
         else \{
             twin.to.left <-(c(0, plist$twins[i,who])[1:sum(who)])
             temp <- cumsum(twin.to.left ==0) #increment if no twin to the left
             # 5 sibs, middle 3 are triplets gives 1,2,2,2,3
             # twin, twin, singleton gives 1,1,2,2,3
             tcount <- table(temp)
             target <- rep(tapply(plist$pos[i,who], temp, mean), tcount)
             \}
         yy <- rep(i, sum(who))
         segments(plist$pos[i,who], yy, target, yy-legh)
                   
         ## draw midpoint MZ twin line
         if (any(plist$twins[i,who] ==1)) \{
           who2 <- which(plist$twins[i,who] ==1)
           temp1 <- (plist$pos[i, who][who2] + target[who2])/2
           temp2 <- (plist$pos[i, who][who2+1] + target[who2])/2
             yy <- rep(i, length(who2)) - legh/2
             segments(temp1, yy, temp2, yy)
             \}
 
         # Add a question mark for those of unknown zygosity
         if (any(plist$twins[i,who] ==3)) \{
           who2 <- which(plist$twins[i,who] ==3)
           temp1 <- (plist$pos[i, who][who2] + target[who2])/2
           temp2 <- (plist$pos[i, who][who2+1] + target[who2])/2
             yy <- rep(i, length(who2)) - legh/2
             text((temp1+temp2)/2, yy, '?')
             \}
         
         # Add the horizontal line 
         segments(min(target), i-legh, max(target), i-legh)
 
         # Draw line to parents.  The original rule corresponded to
         #  pconnect a large number, forcing the bottom of each parent-child
         #  line to be at the center of the bar uniting the children.
         if (diff(range(target)) < 2*pconnect) x1 <- mean(range(target))
         else x1 <- pmax(min(target)+ pconnect, pmin(max(target)-pconnect, 
                                                     parentx))
         y1 <- i-legh
         if(branch == 0)
             segments(x1, y1, parentx, (i-1) + boxh/2)
         else \{
             y2 <- (i-1) + boxh/2
             x2 <- parentx
             ydelta <- ((y2 - y1) * branch)/2
             segments(c(x1, x1, x2), c(y1, y1 + ydelta, y2 - ydelta), 
                      c(x1, x2, x2), c(y1 + ydelta, y2 - ydelta, y2))
             \}
         \}
     \}
\end{nwchunk}

The last set of lines are dotted arcs that connect mulitiple instances of
a subject on the same line.  These instances may or may not be on the
same line.
The arrcconect function draws a quadratic arc between locations $(x_1, y_1)$
and $(x_2, y_2$) whose height is 1/2 unit above a straight line connection.
\begin{nwchunk}
\nwhypb{pedplot-lines4}{pedplot-lines}{pedplot-lines3}=
 arcconnect <- function(x, y) \{
     xx <- seq(x[1], x[2], length = 15)
     yy <- seq(y[1], y[2], length = 15) + (seq(-7, 7))^2/98 - .5
     lines(xx, yy, lty = 2)
     \}
 
 uid <- unique(plist$nid)
 ## JPS 4/27/17: unique above only applies to rows
 ## unique added to for loop iterator
 for (id in unique(uid[uid>0])) \{
     indx <- which(plist$nid == id)
     if (length(indx) >1) \{   #subject is a multiple
         tx <- plist$pos[indx]
         ty <- ((row(plist$pos))[indx])[order(tx)]
         tx <- sort(tx)
         for (j in 1:(length(indx) -1))
             arcconnect(tx[j + 0:1], ty[j+  0:1])
         \}
     \}
\end{nwchunk}

\subsection{Final output}
Remind the user of subjects who did not get
plotted; these are ususally subjects who are married in but without
children.  Unless the pedigree contains spousal information the
routine does not know who is the spouse.
Then restore the plot parameters.  This would only not be done if someone
wants to further annotate the plot.
Last, we give a list of the plot positions for each subject.  Someone
who is plotted twice will have their first position listed.
\begin{nwchunk}
\nwhypb{pedplot-finish2}{pedplot-finish}{pedplot-finish1}=
 ckall <- x$id[is.na(match(x$id,x$id[plist$nid]))]
 if(length(ckall>0)) cat('Did not plot the following people:',ckall,'{\textbackslash}n')
     
 if(!keep.par) par(oldpar)
 
 tmp <- match(1:length(x$id), plist$nid)
 invisible(list(plist=plist, x=plist$pos[tmp], y= row(plist$pos)[tmp],
                boxw=boxw, boxh=boxh, call=Call))        
\end{nwchunk}
\subsection{Symbols}
There are four sumbols corresponding to the four sex codes: square = male,
circle = female, diamond= unknown, and triangle = terminated.  
They are shaded according to the value(s) of affected status for each
subject, where 0=unfilled and 1=filled, and filling uses the standard
arguments of the \Verb!polygon! function.
The nuisance is when the affected status is a matrix, in which case the
symbol will be divided up into sections, clockwise starting at the 
lower left. 
I asked Beth about this (original author) and there was no particular
reason to start at 6 o'clock, but it's now established as history.

The first part of the code is to create the collection of polygons that
will make up the symbol.  These are then used again and again.
The collection is kept as a list with the four elements square, circle,
diamond and triangle.  
Each of these is in turn a list with ncol(affected) element, and each
of those in turn a list of x and y coordinates.
There are 3 cases: the affected matrix has
only one column, partitioning a circle for multiple columns, and 
partitioning the other cases for multiple columns.

\begin{nwchunk}
\nwhyp{pedplot-drawbox2}{pedplot-drawbox}{pedplot-drawbox1}{pedplot-drawbox3}=
 \nwhypf{pedplot-circfun1}{pedplot-circfun}{pedplot-circfun2}
 \nwhypf{pedplot-polyfun1}{pedplot-polyfun}{pedplot-polyfun2}
 if (ncol(affected)==1) \{
     polylist <- list(
         square = list(list(x=c(-1, -1, 1,1)/2,  y=c(0, 1, 1, 0))),
         circle = list(list(x=.5* cos(seq(0, 2*pi, length=50)),
                            y=.5* sin(seq(0, 2*pi, length=50)) + .5)),
         diamond = list(list(x=c(0, -.5, 0, .5), y=c(0, .5, 1, .5))),
         triangle= list(list(x=c(0, -.56, .56),  y=c(0, 1, 1))))
     \}
 else \{
     nc <- ncol(affected)
     square <- polyfun(nc, list(x=c(-.5, -.5, .5, .5), y=c(-.5, .5, .5, -.5),
                                 theta= -c(3,5,7,9)* pi/4))
     circle <- circfun(nc)
     diamond <-  polyfun(nc, list(x=c(0, -.5, 0, .5), y=c(-.5, 0, .5,0),
                                 theta= -(1:4) *pi/2))
     triangle <- polyfun(nc, list(x=c(-.56, .0, .56), y=c(-.5, .5, -.5),
                                  theta=c(-2, -4, -6) *pi/3))
     polylist <- list(square=square, circle=circle, diamond=diamond, 
                      triangle=triangle)
     \}
\end{nwchunk}

The circle function is quite simple.  The number of segments is arbitrary,
50 seems to be enough to make the eye happy.  We draw the ray from 0 to
the edge, then a portion of the arc.  The polygon function will connect
back to the center.
\begin{nwchunk}
\nwhypb{pedplot-circfun2}{pedplot-circfun}{pedplot-circfun1}=
 circfun <- function(nslice, n=50) \{
     nseg <- ceiling(n/nslice)  #segments of arc per slice
     
     theta <- -pi/2 - seq(0, 2*pi, length=nslice +1)
     out <- vector('list', nslice)
     for (i in 1:nslice) \{
         theta2 <- seq(theta[i], theta[i+1], length=nseg)
         out[[i]]<- list(x=c(0, cos(theta2)/2),
                         y=c(0, sin(theta2)/2) + .5)
         \}
     out
     \}
\end{nwchunk}

Now for the interesting one --- dividing a polygon into ``pie slices''.
In computing this we can't use the usual $y= a + bx$ formula for a line,
because it doesn't work for vertical ones (like the sides of the square).
Instead we use the alternate formulation in terms of a dummy variable 
$z$.
\begin{eqnarray*}
  x &=& a + bz \\
  y &=& c + dz \\
\end{eqnarray*}
Furthermore, we choose the constants $a$, $b$, $c$, and $d$ so that 
the side of our polygon correspond to $0 \le z \le 1$.
The intersection of a particular ray at angle theta with a 
particular side will satisfy
\begin{eqnarray}
  theta &=& y/x = \frac{a + bz}{c+dz} \nonumber \\
  z &=& \frac{a\theta -c}{b - d\theta} \label{eq:z} \\
\end{eqnarray}

Equation \ref{eq:z} will lead to a division by zero if the ray from the
origin does not intersect a side, e.g., a vertical divider will be parallel
to the sides of a square symbol.  The only solutions we want have
$0 \le z \le 1$ and are in the `forward' part of the ray.  This latter  %'`
is true if the inner product $x \cos(\theta) + y \sin(\theta) >0$.
Exactly one of the polygon sides will satisfy both conditions.

\begin{nwchunk}
\nwhyp{pedplot-polyfun2}{pedplot-polyfun}{pedplot-polyfun1}{pedplot-polyfun3}=
 polyfun <- function(nslice, object) \{
     # make the indirect segments view
     zmat <- matrix(0,ncol=4, nrow=length(object$x))
     zmat[,1] <- object$x
     zmat[,2] <- c(object$x[-1], object$x[1]) - object$x
     zmat[,3] <- object$y
     zmat[,4] <- c(object$y[-1], object$y[1]) - object$y
 
     # Find the cutpoint for each angle
     #   Yes we could vectorize the loop, but nslice is never bigger than
     # about 10 (and usually <5), so why be obscure?
     ns1 <- nslice+1
     theta <- -pi/2 - seq(0, 2*pi, length=ns1)
     x <- y <- double(ns1)
     for (i in 1:ns1) \{
         z <- (tan(theta[i])*zmat[,1] - zmat[,3])/
             (zmat[,4] - tan(theta[i])*zmat[,2])
         tx <- zmat[,1] + z*zmat[,2]
         ty <- zmat[,3] + z*zmat[,4]
         inner <- tx*cos(theta[i]) + ty*sin(theta[i])
         indx <- which(is.finite(z) & z>=0 &  z<=1 & inner >0)
         x[i] <- tx[indx]
         y[i] <- ty[indx]
         \}
\end{nwchunk}

Now I have the $x,y$ coordinates where each radial slice (the cuts you
would make when slicing a pie) intersects the polygon.  
Add the original vertices of the polygon to the list, sort by angle, and
create the output.  The radial lines are labeled 1,2, \ldots, nslice +1
(the original cut from the center to 6 o'clock is repeated at the end),   %'
and the inserted vertices with a zero.
\begin{nwchunk}
\nwhypb{pedplot-polyfun3}{pedplot-polyfun}{pedplot-polyfun2}=
     nvertex <- length(object$x)
     temp <- data.frame(indx = c(1:ns1, rep(0, nvertex)),
                        theta= c(theta, object$theta),
                        x= c(x, object$x),
                        y= c(y, object$y))
     temp <- temp[order(-temp$theta),]
     out <- vector('list', nslice)
     for (i in 1:nslice) \{
         rows <- which(temp$indx==i):which(temp$indx==(i+1))
         out[[i]] <- list(x=c(0, temp$x[rows]), y= c(0, temp$y[rows]) +.5)
         \}
     out
     \}   
\end{nwchunk}

Finally we get to the drawbox function itself, which is fairly simple.
Updates by JPS in 5/2011 to allow missing, and to fix up shadings and borders.
For affected=0, don't fill.
For affected=1, fill with density-lines and angles.
For affected=-1 (missing), fill with ``?'' in the midpoint of the polygon,
with a size adjusted by how many columns in affected.
For all shapes drawn, make the border the color for the person.

\begin{nwchunk}
\nwhypb{pedplot-drawbox3}{pedplot-drawbox}{pedplot-drawbox2}=
 
   drawbox<- function(x, y, sex, affected, status, col, polylist,
             density, angle, boxw, boxh) \{
         for (i in 1:length(affected)) \{
             if (affected[i]==0) \{
                 polygon(x + (polylist[[sex]])[[i]]$x *boxw,
                         y + (polylist[[sex]])[[i]]$y *boxh,
                         col=NA, border=col)
                 \}
             
             if(affected[i]==1) \{
               ## else \{
               polygon(x + (polylist[[sex]])[[i]]$x * boxw,
                       y + (polylist[[sex]])[[i]]$y * boxh,
                       col=col, border=col, density=density[i], angle=angle[i])            
             \}
             if(affected[i] == -1) \{
               polygon(x + (polylist[[sex]])[[i]]$x * boxw,
                       y + (polylist[[sex]])[[i]]$y * boxh,
                       col=NA, border=col)
               
               midx <- x + mean(range(polylist[[sex]][[i]]$x*boxw))
               midy <- y + mean(range(polylist[[sex]][[i]]$y*boxh))
              
               points(midx, midy, pch="?", cex=min(1, cex*2/length(affected)))
             \}
             
           \}
         if (status==1) segments(x- .6*boxw, y+1.1*boxh, 
                                 x+ .6*boxw, y- .1*boxh,)
         ## Do a black slash per Beth, old line was
         ##        x+ .6*boxw, y- .1*boxh, col=col)
       \}
 
\end{nwchunk}

\subsection{Subsetting}
This section is still experimental and might change.  

Sometimes a pedigree is too large to fit comfortably on one page.
The \Verb!subregion! argument allows one to plot only a portion of the
pedigree based on the plot region.  Along with other tools to
select portions of the pedigree based on relatedness, such as all
the descendents of a particular marriage, it gives a tool for
addressing this.  This breaks our original goal of completely
automatic plots, but users keep asking for more.

The argument is \Verb!subregion=c(min x, max x, min depth, max depth)!,
and works by editing away portions of the \Verb!plist! object
returned by align.pedigree. 
First decide what lines to keep. 
Then take subjects away from each line, 
update spouses and twins,
and fix up parentage for the line below.

JPS 5/23/2011 note:
Found the subregion option to re-scale the y-axis very well, but 
not the x-axis.

\begin{nwchunk}
\nwhypb{pedplot-subregion2}{pedplot-subregion}{pedplot-subregion1}=
 subregion2 <- function(plist, subreg) \{
     if (subreg[3] <1 || subreg[4] > length(plist$n)) 
         stop("Invalid depth indices in subreg")
     lkeep <- subreg[3]:subreg[4]
     for (i in lkeep) \{
         if (!any(plist$pos[i,]>=subreg[1] & plist$pos[i,] <= subreg[2]))
             stop(paste("No subjects retained on level", i))
         \}
     
     nid2 <- plist$nid[lkeep,]
     n2   <- plist$n[lkeep]
     pos2 <- plist$pos[lkeep,]
     spouse2 <- plist$spouse[lkeep,]
     fam2 <- plist$fam[lkeep,]
     if (!is.null(plist$twins)) twin2 <- plist$twins[lkeep,]
     
     for (i in 1:nrow(nid2)) \{
         keep <- which(pos2[i,] >=subreg[1] & pos2[i,] <= subreg[2])
         nkeep <- length(keep)
         n2[i] <- nkeep
         nid2[i, 1:nkeep] <- nid2[i, keep]
         pos2[i, 1:nkeep] <- pos2[i, keep]
         spouse2[i,1:nkeep] <- spouse2[i,keep]
         fam2[i, 1:nkeep] <- fam2[i, keep]
         if (!is.null(plist$twins)) twin2[i, 1:nkeep] <- twin2[i, keep]
 
         if (i < nrow(nid2)) \{  #look ahead
             tfam <- match(fam2[i+1,], keep, nomatch=0)
             fam2[i+1,] <- tfam
             if (any(spouse2[i,tfam] ==0)) 
                 stop("A subregion cannot separate parents")
             \}
         \}
     
     n <- max(n2)
     out <- list(n= n2[1:n], nid=nid2[,1:n, drop=F], pos=pos2[,1:n, drop=F],
                 spouse= spouse2[,1:n, drop=F], fam=fam2[,1:n, drop=F])
     if (!is.null(plist$twins)) out$twins <- twin2[, 1:n, drop=F]
     out
     \}
\end{nwchunk}


\subsection{Legends}

We define a function to draw a legend for the affected matrix. We do so
by making use of the pie() function, which will draw a circle that will look
like a woman (circle) in the pedigree who has all affected indicators ==1.  
We do not show what the ``?'' means, and we do not cover what colors are 
indicated by the coloring applied to subjects.

We allow the legend to be added to the current pedigree plot by default,
and it also works to draw a legend on a separate page.  The {\em new} argument
controls this option. When new=TRUE, the default, the plot is added to the 
current plot (assumed a pedigree plot), and placed in one of the corners
of the plot given by {\em location}, which has options "bottomright", 
"topright", "topleft", and "bottomleft", with ``bottomright'' the default.

If new=FALSE, the pie graph is plotted from (-1,1) for both x and y, centered 
at 0,0 with radius 1. With angle.init=90 and twopi = 2*pi, we control the 
start to be at the top and the sections are plotted counter-clockwise, respectively, which are some of the settings from the original pie() function.  

When we adapted the pie() function to plot in different, non-(0,0) locations
on the pedigree, we had these major issues:

1) The Y-axis actually goes from min(y) at the top and max(y) at the bottom.
2) To get the polygon in pie() to not be oblong, we made sure to use asp=1, 
which re-sets the x- and/or y-axis again.  Therefore, we have to manage the 
placing of the pie in reference to those updated scalings using par(``usr'').
3) We have to choose a center that is not 0,0, and have to add the center
x,y coordinates to some of the default settings of pie().

We carry forward from the plot.pedigree the same density and angle defaults
for shading sections of each subject's symbol with polygon.  


\begin{nwchunk}
\nwhypn{pedigree.legend}=
 
 pedigree.legend <- function (ped, labels = dimnames(ped$affected)[[2]],
     edges = 200, radius=NULL, location="bottomright", new=TRUE,
     density=c(-1, 35,65,20),  angle = c(90, 65, 40, 0), ...) 
 \{
    
     naff <- max(ncol(ped$affected),1)
 
     x <- rep(1,naff)
     
     # Defaults for plotting on separate page:
     ## start at the top, always counter-clockwise, black/white
     init.angle <- 90
     twopi <- 2 * pi
     col <- 1
 
     default.labels <- paste("affected-", 1:naff, sep='')
     if (is.null(labels)) labels <- default.labels
     
     ## assign labels to those w/ zero-length label
     whichNoLab <- which(nchar(labels) < 1)
     if(length(whichNoLab))
       labels[whichNoLab] <- paste("affected-", whichNoLab, sep='')
 
     
     x <- c(0, cumsum(x)/sum(x))
     dx <- diff(x)
     nx <- length(dx)
     ## settings for plotting on a new page
     if(!new) \{
       plot.new()
       
       pin <- par("pin")
       # radius, xylim, center, line-lengths set to defaults of pie()
       radius <- 1
       xlim <- ylim <- c(-1, 1)
       center <- c(0,0)
       llen <- 0.05
       
       if (pin[1L] > pin[2L]) 
         xlim <- (pin[1L]/pin[2L]) * xlim
       else ylim <- (pin[2L]/pin[1L]) * ylim
       
       plot.window(xlim, ylim, "", asp = 1)
       
     \} else \{
       ## Settings to add to pedigree plot
       ## y-axis is flipped, so adjust angle and rotation
       init.angle <- -1*init.angle
       twopi <- -1*twopi
 
       ## track usr xy limits. With asp=1, it re-scales to have aspect ratio 1:1
       usr.orig <- par("usr")
       plot.window(xlim=usr.orig[1:2], ylim=usr.orig[3:4], "", asp=1)
       usr.asp1 <- par("usr")
      
       ## also decide on good center/radius if not given
       if(is.null(radius))
         radius <- .5
       
       ## set line lengths
       llen <- radius*.15
       
       ## get center of pie chart for coded
       pctusr <- .10*abs(diff(usr.asp1[3:4]))
       center = switch(location,
         "bottomright" = c(max(usr.asp1[1:2])-pctusr,max(usr.asp1[3:4])-pctusr),
         "topright" = c(max(usr.asp1[1:2])-pctusr, min(usr.asp1[3:4]) + pctusr),
         "bottomleft" =c(min(usr.asp1[1:2]) + pctusr, max(usr.asp1[3:4])-pctusr),
         "topleft" = c(min(usr.asp1[1:2]) + pctusr, min(usr.asp1[3:4]) + pctusr))
      
     \}
     
     col <- rep(col, length.out = nx)
     border <- rep(1, length.out = nx)
     lty <- rep(1, length.out = nx)
     angle <- rep(angle, length.out = nx)
     density <- rep(density, length.out = nx)
   
     t2xy <- function(t) \{
         t2p <- twopi * t + init.angle * pi/180
         list(x = radius * cos(t2p), y = radius * sin(t2p))
     \}
     for (i in 1L:nx) \{
         n <- max(2, floor(edges * dx[i]))
         P <- t2xy(seq.int(x[i], x[i + 1], length.out = n))
         P$x <- P$x + center[1]
         P$y <- P$y + center[2]
         
         polygon(c(P$x, center[1]), c(P$y, center[2]), density = density[i],
                 angle = angle[i], border = border[i], col = col[i],
                 lty = lty[i])
 
         P <- t2xy(mean(x[i + 0:1]))
         if(new) \{
           ## not centered at 0,0, so added center to x,y
           P$x <- P$x + center[1]
           P$y <- center[2] + ifelse(new, P$y, -1*P$y)
         \}
         
         lab <- as.character(labels[i])
         if (!is.na(lab) && nzchar(lab)) \{
           ## put lines
           lines(x=c(P$x, P$x + ifelse(P$x<center[1], -1*llen, llen)),
                 y=c(P$y, P$y + ifelse(P$y<center[2], -1*llen, llen)))
 
           ##  put text just beyond line-length away from pie
           text(x=P$x + ifelse(P$x < center[1], -1.2*llen, 1.2*llen),
                y=P$y + ifelse(P$y < center[2], -1.2*llen, 1.2*llen),
                labels[i], xpd = TRUE, 
                adj = ifelse(P$x < center[1], 1, 0), ...)
         \}
     \}
     
     invisible(NULL)
 \}
\end{nwchunk}






\section{Intro to Pedigree Shrink}

The pedigree.shrink functions were initially written to deal with a pedigree
represented as a data.frame with pedTrim, written by Steve Iturria, to trim 
the subjects from a pedigree who were less useful for linkage and family 
association studies.  It was later turned into a package called pedShrink 
by Daniel Schaid's group, still working on a pedigree, but assuming it was 
just a data.frame.  Later, the functions were managed by Jason Sinnwell who 
worked with the 2010 version of the pedigree object by Terry Therneau in 
planning to group many of the pedigree functions together into an enhanced 
kinship package.

This file also contains the pedigree.unrelated function, developed by Dan 
Schaid and Shannon McDonnell, which uses the kinship matrix to 
determine relatedness of subjects in a pedigree, and returns the person id
of one of the maximal sets of individuals that are not related. 
Details described below.


\section{Pedigree Shrink}
The pedigree.shrink function trims an object of class pedigree, and 
returns a list with information about how the pedigree was shrunk, 
and the final shrunken pedigree object.

\emph{pedigree.shrink}.  
Accepts the following input
\begin{description}
  \item[ped] a pedigree object
  \item[avail] indicator vector of availability of each person in the pedigree
  \item[seed] seed to control randomness
  \item[maxBits] bit size to shrink the pedigree size under
\end{description}

\begin{nwchunk}
\nwhypn{pedigree.shrink}=
 
 pedigree.shrink <- function(ped, avail, affected=NULL, seed=NULL, maxBits = 16)\{
   if(class(ped) != "pedigree")
     stop("Must be a pegigree object.{\textbackslash}n")
   
   ##set the seed for random selections
 
   if(is.null(seed))  \{
     seed <- sample(2^20, size=1)
   \}
   set.seed(seed)
   
   if(any(is.na(avail)))
     stop("NA values not allowed in avail vector.")
   
   if(is.null(affected))
     affected = if(is.matrix(ped$affected)) ped$affected[,1] else ped$affected
   
   ped$affected = affected
   
   idTrimmed <- numeric()
   idList <- list()
   nOriginal <- length(ped$id)
   
   bitSizeOriginal <- bitSize(ped)$bitSize
   
   ## first find unavailable subjects to remove anyone who is not 
   ## available and does not have an available descendant
   
   idTrimUnavail <- findUnavailable(ped, avail)
   
   if(length(idTrimUnavail)) \{    
     
     pedTrimmed <- pedigree.trim(idTrimUnavail, ped)
     avail <- avail[match(pedTrimmed$id, ped$id)]
     idTrimmed <- c(idTrimmed, idTrimUnavail)
     idList$unavail <- paste(idTrimUnavail, collapse=' ')
     
   \} else \{
   ## no trimming, reset to original ped
     pedTrimmed <- ped
   \}
   
   ## Next trim any available terminal subjects with unknown phenotype
   ## but only if both parents are available
   
   ## added nNew>0 check because no need to trim anymore if empty ped
   
   nChange <- 1
   idList$noninform = NULL
   nNew <- length(pedTrimmed$id)
   
   while(nChange > 0 & nNew > 0)\{
     nOld <- length(pedTrimmed$id)
     
     ## findAvailNonInform finds non-informative, but after suggesting 
     ## their removal, checks for more unavailable subjects before returning
     idTrimNonInform <- findAvailNonInform(pedTrimmed, avail)
     
     if(length(idTrimNonInform)) \{
       pedNew <- pedigree.trim(idTrimNonInform, pedTrimmed)
       avail <- avail[match(pedNew$id, pedTrimmed$id)]
       idTrimmed <- c(idTrimmed, idTrimNonInform)
       idList$noninform = paste(c(idList$noninform, 
                                  idTrimNonInform), collapse=' ')
       pedTrimmed <- pedNew
       
     \}
     nNew <- length(pedTrimmed$id)
     nChange <- nOld - nNew
     
   \}
   
   ##  Determine number of subjects & bitSize after initial trimming
   nIntermed <- length(pedTrimmed$id)
   
   bitSize <- bitSize(pedTrimmed)$bitSize
   
   ## Now sequentially shrink to fit bitSize <= maxBits
   
   bitVec <- c(bitSizeOriginal,bitSize)
   
   isTrimmed <- TRUE
   idList$affect=NULL 
   
   while(isTrimmed & (bitSize > maxBits))
   \{  
     
     ## First, try trimming by unknown status
     save <- findAvailAffected(pedTrimmed, avail, affstatus=NA)
     isTrimmed <- save$isTrimmed
     
     ## Second, try trimming by unaffected status if no unknowns to trim
     if(!isTrimmed)
     \{
       save <- findAvailAffected(pedTrimmed, avail, affstatus=0)
       isTrimmed <- save$isTrimmed
       
     \}
     
     
    ## Third, try trimming by affected status if no unknowns & no unaffecteds
    ## to trim
     if(!isTrimmed) \{
       save <- findAvailAffected(pedTrimmed, avail, affstatus=1)
       isTrimmed <- save$isTrimmed
     \}
     
     if(isTrimmed)  \{
       pedTrimmed <- save$ped
       avail <- save$newAvail
       bitSize <- save$bitSize
       bitVec <- c(bitVec, bitSize)          
       idTrimmed <- c(idTrimmed, save$idTrimmed)
       idList$affect = paste(c(idList$affect, save$idTrimmed), 
                             collapse=' ')
     \}
    
     
   \} 
     ## end while (isTrimmed) & (bitSize > maxBits)
 
   nFinal <- length(pedTrimmed$id)
   
   obj <- list(pedObj = pedTrimmed,
               idTrimmed = idTrimmed,
               idList = idList,
               bitSize = bitVec,
               avail=avail,
               pedSizeOriginal = nOriginal,
               pedSizeIntermed = nIntermed,
               pedSizeFinal  = nFinal,
               seed = seed)
   
   oldClass(obj) <- "pedigree.shrink"
   
   return(obj)
 \} 
 
\end{nwchunk}
  
  
\subsection{Sub-Functions}


These next functions were written to support pedigree.shrink.
In making the new kinship2 package to include pedigree.shrink, Jason Sinnwell
decided to add functionality to removed  subjects from a pedigree object 
given their id.  Then within pedigree.shrink, any removal of subjects consists
of two steps, identifying who to remove by their ids. Then removing them with
a new pedigree.trim function.  

The problem with pedigree.trim is that if the removal of any subject causes
a marriage to be split and have parentless children, it will cause a problem.

Therefore, when using functions like findAvalNonInform and findAvalAffected
for persons to remove, follow them up with a call findUnavailable, after 
setting the removal candidates availability to FALSE, so clear up any 
removals.

This last step was re-written by Jason Sinnwell on 6/1/2011, and his test cases
seemed to test against the results before the re-write. He expects there to 
be bugs to be discovered down the road.


What was previously pedTrim is now split into two functions, pedigree.trim and findUnavail.  

pedigree.trim : remove subjects from pedigree object given their id. 
Update for version 1.2.8 (9/27/11) Allow creation of an empty pedigree 
if all IDs are removed. This allows bitSize and 
pedigree.shrink to still complete with an empty pedigree. 

findUnavail: identify subjects are not available and who do not have 
an available descendant.  Do this iteratively by successively removing 
unavailable terminal nodes.  Written by  Steve Iturria, PhD, modified 
by Dan Schaid.

\begin{nwchunk}
\nwhypn{pedigree.trim}=
 
 pedigree.trim <- function(removeID, ped)\{
 ## trim subjects from a pedigree who match the removeID 
 ## trim relation matrix as well
 
 if(class(ped) != "pedigree")
 stop("Must be a pegigree object.{\textbackslash}n")
 
 rmidx <- match(removeID, ped$id)
 if(length(rmidx)>0) \{
 pedtrimmed <- ped[-rmidx]
 return(pedtrimmed)
 \} else \{
 return(ped)
 \}
 \}
 
\end{nwchunk}

Place the two exclude functions within the same file as findUnavailable
because that is the only place they are used. Pretty self-documenting.

\begin{nwchunk}
\nwhypn{findUnavailable}=
 
 findUnavailable <-function(ped, avail) \{
 
   ## find id within pedigree anyone who is not available and
   ## does not have an available descendant
 
   ## avail = TRUE/1 if available, FALSE/0 if not
 
   ## will do this iteratively by successively removing unavailable
   ## terminal nodes
   ## Steve Iturria, PhD, modified by Dan Schaid
 
 cont <- TRUE                  # flag for whether to keep iterating
 
 is.terminal <- (is.parent(ped$id, ped$findex, ped$mindex) == FALSE)
   ## JPS 3/10/14 add strings check in case of char ids
 pedData <- data.frame(id=ped$id, father=ped$findex, mother=ped$mindex,
 sex=ped$sex, avail, is.terminal, stringsAsFactors=FALSE)  
 iter <- 1
 
 while(cont)  \{
   ##print(paste("Working on iter", iter))
 
 num.found <- 0
 idx.to.remove <- NULL
 
 for(i in 1:nrow(pedData))
 \{
   
   if(pedData$is.terminal[i])
   \{
   if( pedData$avail[i] == FALSE )   # if not genotyped         
   \{
   idx.to.remove <- c(idx.to.remove, i)
   num.found <- num.found + 1
   
   ## print(paste("  removing", num.found, "of", nrow(pedData)))
   \}
   \}
   
 \}
 
 if(num.found > 0) \{
 
 pedData <- pedData[-idx.to.remove, ]
   ## re-index parents, which varies depending on if the removed indx is
   ## prior to parent index
 for(k in 1:nrow(pedData))\{
 if(pedData$father[k] > 0) \{
 pedData$father[k] <- pedData$father[k] -
 sum(idx.to.remove < pedData$father[k])
 \}
 if(pedData$mother[k]+0) \{
 pedData$mother[k] <- pedData$mother[k] -
 sum(idx.to.remove < pedData$mother[k])
 \}
 \}
 pedData$is.terminal <-
 (is.parent(pedData$id, pedData$father, pedData$mother) == FALSE)
 
 \}
 else \{
 cont <- FALSE
 \}
 iter <- iter + 1   
 
 \}
 
   ## A few more clean up steps
 
   ## remove unavailable founders
 tmpPed <- excludeUnavailFounders(pedData$id, 
 pedData$father, pedData$mother, pedData$avail)
 
 tmpPed <- excludeStrayMarryin(tmpPed$id, tmpPed$father, tmpPed$mother)
 
 id.remove <- ped$id[is.na(match(ped$id, tmpPed$id))]
 
 return(id.remove)
 
 \}
 
 excludeStrayMarryin <- function(id, father, mother)\{
   # get rid of founders who are not parents (stray available marryins
   # who are isolated after trimming their unavailable offspring)
   ## JPS 3/10/14 add strings check in case of char ids
 trio <- data.frame(id=id, father=father, mother=mother, stringsAsFactors=FALSE)
 parent <- is.parent(id, father, mother)
 founder <- is.founder(father, mother)
 
 exclude <- !parent & founder
 trio <- trio[!exclude,,drop=FALSE]
 return(trio)
 
 \}
 
 excludeUnavailFounders <- function(id, father, mother, avail)
 \{
   nOriginal <- length(id)
   idOriginal <- id   
   zed <- father!=0 & mother !=0
   ## concat ids to represent marriages. 
   ## Bug if there is ":" in char subj ids
   marriage <- paste(id[father[zed]], id[mother[zed]], sep=":" )
   
   sibship <- tapply(marriage, marriage, length)
   nm <- names(sibship)
   
   splitPos <- regexpr(":",nm)
   dad <- substring(nm, 1, splitPos-1)
   mom <- substring(nm, splitPos+1,  nchar(nm))
   
   ##  Want to look at parents with only one child.
   ##  Look for parents with > 1 marriage.  If any
   ##  marriage has > 1 child then skip this mom/dad pair.
   
   nmarr.dad <- table(dad)
   nmarr.mom <- table(mom)
   skip <- NULL
   
   if(any(nmarr.dad > 1)) \{
   ## Dads in >1 marriage
   ckdad <- which(as.logical(match(dad,
   names(nmarr.dad)[which(nmarr.dad > 1)],nomatch=FALSE)))
   skip <- unique(c(skip, ckdad))
   \}
   
   if(any(nmarr.mom > 1)) \{
   ## Moms in >1 marriage
   ckmom <- which(as.logical(match(mom,
   names(nmarr.mom)[which(nmarr.mom > 1)],nomatch=FALSE)))
   skip <- unique(c(skip, ckmom))
   \}
   
   if(length(skip) > 0) \{
   dad <- dad[-skip]
   mom <- mom[-skip]
   zed <- (sibship[-skip]==1) 
   \} else \{
   zed <- (sibship==1)
   \}
   
   n <- sum(zed)
   idTrimmed <- NULL
   if(n>0)
   \{
   
   # dad and mom are the parents of sibships of size 1
   dad <- dad[zed]
   mom <- mom[zed]
   for(i in 1:n)\{
   ## check if mom and dad are founders (where their parents = 0)
   dad.founder <- (father[id==dad[i]] == 0) & (mother[id==dad[i]] == 0)
   mom.founder <- (father[id==mom[i]] == 0) & (mother[id==mom[i]] == 0)
   both.founder <- dad.founder & mom.founder
   
   ## check if mom and dad have avail
   dad.avail <- avail[id==dad[i]]
   mom.avail <- avail[id==mom[i]]
   
   ## define not.avail = T if both mom & dad not avail
   not.avail <- (dad.avail==FALSE & mom.avail==FALSE)
   
   if(both.founder & not.avail)   \{
   ## remove mom and dad from ped, and zero-out parent 
   ## ids of their child
   
   child <- which(father==which(id==dad[i]))          
   father[child] <- 0
   mother[child] <- 0
   
   idTrimmed <- c(idTrimmed, dad[i], mom[i])
   
   excludeParents <- (id!=dad[i]) & (id!=mom[i])
   id <- id[excludeParents]
   father <- father[excludeParents]
   mother <- mother[excludeParents]
   
   ## re-index father and mother, assume len(excludeParents)==2
   father <- father - 1*(father > which(!excludeParents)[1]) -
   1*(father > which(!excludeParents)[2])
   
   mother <- mother - 1*(mother > which(!excludeParents)[1]) -
   1*(mother > which(!excludeParents)[2])
   
   avail <- avail[excludeParents]
   \} 
   \}
   \}
   
   nFinal <- length(id)
   nTrimmed = nOriginal - nFinal 
   
   return(list(nTrimmed = nTrimmed, idTrimmed=idTrimmed,
   id=id, father=father, mother=mother))
 \}
 
\end{nwchunk}

Function to calculate pedigree bit size, which is 
2 * n.NonFounder  - n.Founder.  It is an indicator for how much resources
the pedigree will require to be processed by linkage algorithms to calculate
the likelihood of the observed genotypes given the pedigree structure.

The Lander-Green handles smaller pedigrees and many markers
The Elston-Stewart handles larger pedigrees and fewer markers.

\begin{nwchunk}
\nwhypn{bitSize}=
  ## renamed from pedBits, part of pedigree.shrink functions
 bitSize <- function(ped) \{
   ## calculate bit size of a pedigree
 
 if(class(ped) != "pedigree")
 stop("Must be a pegigree object.{\textbackslash}n")
 
 father = ped$findex
 mother = ped$mindex
 id = ped$id
 
 founder <- father==0 & mother==0
 pedSize <- length(father)
 nFounder <- sum(founder)
 nNonFounder <- pedSize - nFounder
 bitSize <- 2*nNonFounder - nFounder
 return(list(bitSize=bitSize,
 nFounder = nFounder,
 nNonFounder = nNonFounder))
 \}
 
\end{nwchunk}

Two functions to identify subjects to remove by other indicators0
than availability. 

findAvailNonInform: id subjects to remove who are available, but not 
informative. This function was formerly trimAvailNonInform().


findAvailAffected: id subjects to remove who were not removed by 
findUnavailable(), but who would be best to remove given their 
affected status.  Try trimming one subject by with affected matching 
affstatus.  If there are ties of multiple subjects that reduce bit 
size equally, randomly choose one of them. This function was formerly named pedTrimOneSubj().
On 3/10/14, Nick Larson found a bug with char ids when stringsAsFactors was TRUE; this
is now fixed with the option set specifically in the data.frames sset to FALSE.


\begin{nwchunk}
\nwhypn{findAvailNonInform}=
 findAvailNonInform <- function(ped, avail)\{
 
   ## trim persons who are available but not informative b/c not parent
   ## by setting their availability to FALSE, then call findUnavailable()
   ## JPS 3/10/14 add strings check in case of char ids
 pedData <- data.frame(id=ped$id, father=ped$findex, 
 mother=ped$mindex, avail=avail, stringsAsFactors=FALSE )
 
 checkParent <- is.parent(pedData$id, pedData$father, pedData$mother)
 
 for(i in 1:nrow(pedData))\{
 
 if(checkParent[i]==FALSE & avail[i]==TRUE & 
 all(ped$affected[i]==0, na.rm=TRUE)) \{
 
   ## could use ped$affected[i,] if keep matrix
 
 fa <- pedData$id[pedData$father[i]]
 mo <- pedData$id[pedData$mother[i]]
 if(avail[pedData$id==fa] & avail[pedData$id==mo])
 \{
   pedData$avail[i] <- FALSE
 \}
 \}
 \}
 
 idTrim <- findUnavailable(ped, pedData$avail)
 return(idTrim)
 \} 
 
\end{nwchunk}

\begin{nwchunk}
\nwhypn{findAvailAffected}=
 findAvailAffected <- function(ped, avail, affstatus)
   ## Try trimming one subject by affection status indicator
   ## If ties for bits removed, randomly select one of the subjects
 
 \{
   
   notParent <- !is.parent(ped$id, ped$findex, ped$mindex)
   
   if(is.na(affstatus)) \{
   possiblyTrim <- ped$id[notParent & avail & is.na(ped$affected)]
   \} else \{
   possiblyTrim <- ped$id[notParent & avail & ped$affected==affstatus]
   \}
   nTrim <- length(possiblyTrim)
   
   if(nTrim == 0)
   \{
   return(list(ped=ped,
   idTrimmed = NA,
   isTrimmed = FALSE,
   bitSize = bitSize(ped)$bitSize))
  \}
   
   trimDat <- NULL
   
   for(idTrim in possiblyTrim) \{
  
   avail.try <- avail
   avail.try[ped$id==idTrim] <- FALSE
   id.rm <- findUnavailable(ped, avail.try)
   newPed <- pedigree.trim(id.rm, ped)
   trimDat <- rbind(trimDat,
   c(id=idTrim, bitSize=bitSize(newPed)$bitSize))
   \}
   
   bits <- trimDat[,2]
   
   # trim by subject with min bits. This trims fewer subject than
   # using max(bits).
   
   idTrim <- trimDat[bits==min(bits), 1]
   
   ## break ties by random choice
   if(length(idTrim) > 1)
   \{
   rord <- order(runif(length(idTrim)))
   idTrim <- idTrim[rord][1]
   \}
   
   avail[ped$id==idTrim] <- FALSE
   id.rm <- findUnavailable(ped, avail)
   newPed <- pedigree.trim(id.rm, ped)
   pedSize <- bitSize(newPed)$bitSize
   avail <- avail[!(ped$id %in% id.rm)]
   
   return(list(ped=newPed,
   newAvail = avail,
   idTrimmed = idTrim,
   isTrimmed = TRUE,
   bitSize = pedSize))
 \}
 
\end{nwchunk}

Group other functions used in the above main functions
together as pedigree.shrink.minor.R


These functions get indicator vectors of who is a parent, 
founder, or disconnected

\begin{nwchunk}
\nwhypn{pedigree.shrink.minor}=
 
 is.parent <- function(id, findex, mindex)\{
   # determine subjects who are parents
   # assume input of father/mother indices, not ids
 
 father <- mother <- rep(0, length(id))
 father[findex>0] <- id[findex]
 mother[mindex>0] <- id[mindex]
 
 isFather <- !is.na(match(id, unique(father[father!=0])))
 isMother <- !is.na(match(id, unique(mother[mother!=0])))
 isParent <- isFather |isMother
 return(isParent)
 \}
 
 is.founder <- function(mother, father)\{
 check <- (father==0) & (mother==0)
 return(check)
 \}
 
 is.disconnected <- function(id, findex, mindex)
 \{
   
   # check to see if any subjects are disconnected in pedigree by checking for
   # kinship = 0 for all subjects excluding self
   father <- id[findex]
   mother <- id[mindex]  
   kinMat <- kinship(id, father, mother)
   diag(kinMat) <- 0
   disconnected <- apply(kinMat==0.0, 1, all)
   
   return(disconnected)
 \}
 
\end{nwchunk}

Print a pedigree.shrink object.  Tell the original bit size and the trimmed bit size.

\begin{nwchunk}
\nwhypn{print.pedigree.shrink}=
 print.pedigree.shrink <- function(x, ...)\{
 
 printBanner(paste("Shrink of Pedigree ", unique(x$pedObj$ped), sep=""))
 
 cat("Pedigree Size:{\textbackslash}n")
 
 if(length(x$idTrimmed) > 2)
 \{
   n <- c(x$pedSizeOriginal, x$pedSizeIntermed, x$pedSizeFinal)
   b <- c(x$bitSize[1], x$bitSize[2], x$bitSize[length(x$bitSize)])
   row.nms <- c("Original","Only Informative","Trimmed")
 \} else \{
   n <- c(x$pedSizeOriginal, x$pedSizeIntermed)
   b <- c(x$bitSize[1], x$bitSize[2])
   row.nms <- c("Original","Trimmed")
 \}
 
 df <- data.frame(N.subj = n, Bits = b)
 rownames(df) <- row.nms
 print(df, quote=FALSE)
 
 if(!is.null(x$idList$unavail)) 
 cat("{\textbackslash}n Unavailable subjects trimmed:{\textbackslash}n", x$idList$unavail, "{\textbackslash}n")
 
 if(!is.null(x$idList$noninform)) 
 cat("{\textbackslash}n Non-informative subjects trimmed:{\textbackslash}n", x$idList$noninform, "{\textbackslash}n")
 
 if(!is.null(x$idList$affect)) 
 cat("{\textbackslash}n Informative subjects trimmed:{\textbackslash}n", x$idList$affect, "{\textbackslash}n")
 
   ##cat("{\textbackslash}n Pedigree after trimming:", x$bitSize, "{\textbackslash}n")
 
 invisible()
 \}
 
\end{nwchunk}

\begin{nwchunk}
\nwhypn{printBanner}=
 
 printBanner <- function(str, banner.width=options()$width, char.perline=.75*banner.width, border = "=")\{
 
   # char.perline was calculated taking the floor of banner.width/3
 
 vec <- str
 new<-NULL
 onespace<-FALSE
 for(i in 1:nchar(vec))\{
 if (substring(vec,i,i)==' ' && onespace==FALSE)\{
 onespace<-TRUE
 new<-paste(new,substring(vec,i,i),sep="")\}
 else if (substring(vec,i,i)==' ' && onespace==TRUE)
 \{onespace<-TRUE\}
 else\{
 onespace<-FALSE
 new<-paste(new,substring(vec,i,i),sep="")\}
 \}
 
 where.blank<-NULL
 indx <- 1
 
 for(i in 1:nchar(new))\{
 if((substring(new,i,i)==' '))\{
 where.blank[indx]<-i
 indx <- indx+1
 \}
 \}
 
   # Determine the position in the where.blank vector to insert the Nth character position of "new"
 j<-length(where.blank)+1
 
   # Add the Nth character position of the "new" string to the where.blank vector.
 where.blank[j]<-nchar(new)
 
 begin<-1
 end<-max(where.blank[where.blank<=char.perline])
 
   # If end.ok equals NA then the char.perline is less than the position of the 1st blank.
 end.ok <- is.na(end) 
 
   # Calculate a new char.perline. 
 if (end.ok==TRUE)\{ 
 char.perline <- floor(banner.width/2)
 end<-max(where.blank[where.blank<=char.perline])
 \}
 
 cat(paste(rep(border, banner.width), collapse = ""),"{\textbackslash}n")
 
 repeat \{
 titleline<-substring(new,begin,end)
 n <- nchar(titleline)
 if(n < banner.width)
 \{
   n.remain <- banner.width - n
   n.left <- floor(n.remain/2)
   n.right <- n.remain - n.left
   for(i in 1:n.left) titleline <- paste(" ",titleline,sep="")
   for(i in 1:n.right) titleline <- paste(titleline," ",sep="")
   n <- nchar(titleline)
 \}
 cat(titleline,"{\textbackslash}n")
 begin<-end+1
 end.old <- end
   # Next line has a problem when used in R.  Use print.banner.R until fixed.
   # Does max with an NA argument
 tmp <- where.blank[(end.old<where.blank) & (where.blank<=end.old+char.perline+1)]
 if(length(tmp)) end <- max(tmp)
 else break
 
   #   end<-max(where.blank[(end.old<where.blank)&(where.blank<=end.old+char.perline+1)])
   #   end.ok <- is.na(end)
   #   if (end.ok==TRUE)
   #      break
 \}
 
 cat(paste(rep(border, banner.width), collapse = ""), "{\textbackslash}n")
 invisible()
 
 \}
 
\end{nwchunk}

Plot a pedigree.shrink object, which calls the plot.pedigree function on the trimmed 
pedigree object.

\begin{nwchunk}
\nwhypn{plot.pedigree.shrink}=
 
 plot.pedigree.shrink <- function(x, bigped=FALSE, title="", xlegend="topright", ...)\{
 
   ##  Plot pedigrees, coloring subjects according
   ##   to availability, shaded by affected status used in shrink
 
 if (bigped == FALSE) \{
 tmp <- plot(x$pedObj, col = x$avail + 1,keep.par=T)
 \}
 else \{
 tmp <- plot.pedigree(x$pedObj, align = FALSE, packed = FALSE, 
 col = x$avail + 1, cex = 0.5, symbolsize = 0.5,keep.par=T)
 \}
 
 legend(x = xlegend, legend = c("DNA Available", "UnAvailable"), 
 pch = c(1, 1), col = c(2, 1), bty = "n", cex=.5)
 title(paste(title, "{\textbackslash}nbits = ", x$bitSize[length(x$bitSize)]),cex.main=.9)
 invisible(tmp)
 \} 
 
\end{nwchunk}

/section{Pedigree Unrelated}

Purpose: Determine set of maximum number of unrelated
available subjects from a pedigree
PI:      Dan Schaid
Author(s): Dan Schaid, Shannon McDonnell
Dates:   Created: 10/19/2007, Moved to kinship2: 6/2011

In many pedigrees there are multiple sets of subjects that could be of the 
size of the maximal set of unrelated subjects in a pedigree.  The set could
contain a married-in uncle and any of a set of siblings from his 
sister-in-law's family.  Therefore, the maximal sets include the uncle and 
any of the sibship of his wife's sister.

\begin{nwchunk}
\nwhypn{pedigree.unrelated}=
 
   ## Authors: Dan Schaid, Shannon McDonnell
   ## Updated by Jason Sinnwell
 
 pedigree.unrelated <- function(ped, avail) \{
 
   # Requires: kinship function
 
   # Given vectors id, father, and mother for a pedigree structure,
   # and avail = vector of T/F or 1/0 for whether each subject
   # (corresponding to id vector) is available (e.g.,
   # has DNA available), determine set of maximum number
   # of unrelated available subjects from a pedigree.
 
   # This is a greedy algorithm that uses the kinship
   # matrix, sequentially removing rows/cols that
   # are non-zero for subjects that have the most
   # number of zero kinship coefficients (greedy
   # by choosing a row of kinship matrix that has
   # the most number of zeros, and then remove any
   # cols and their corresponding rows that are non-zero.
   # To account for ties of the count of zeros for rows,
   # a random choice is made. Hence, running this function
   # multiple times can return different sets of unrelated
   # subjects.
 
   id <- ped$id
   avail <- as.integer(avail)
   
   kin <- kinship(ped)
   
   ord <- order(id)
   id <- id[ord]
   avail <- as.logical(avail[ord])
   kin <- kin[ord,][,ord]
   
   rord <- order(runif(nrow(kin)))
   
   id <- id[rord]
   avail <- avail[rord]
   kin <- kin[rord,][,rord]
   
   id.avail <- id[avail]
   kin.avail <- kin[avail,,drop=FALSE][,avail,drop=FALSE]
   
   diag(kin.avail) <- 0
   
   while(any(kin.avail > 0))  \{
     nr <- nrow(kin.avail)
     indx <- 1:nrow(kin.avail)
     zero.count <- apply(kin.avail==0, 1, sum)
     
     mx <- max(zero.count[zero.count < nr])
     mx.zero <- indx[zero.count == mx][1]
     
     exclude <- indx[kin.avail[, mx.zero] > 0]
     
     kin.avail <- kin.avail[- exclude, , drop=FALSE][, -exclude, drop=FALSE]
     
   \}
   
   choice <- sort(dimnames(kin.avail)[[1]])
   
   return(choice)
 \}
 
\end{nwchunk}
\section{Checks}
Last are various helper routines and data checks.
\subsection{kindepth}
One helper function used throughout computes the depth of
each subject in the pedigree.  
For each subject this is defined as the maximal number of
generations of ancestors: how far to the farthest
founder.  
This can be called with a pedigree object, or with the 
full argument list.  In the former case we can simply
skip a step.
\begin{nwchunk}
\nwhypn{kindepth}=
 kindepth <- function(id, dad.id, mom.id, align=FALSE) \{
     if (class(id)=='pedigree' || class(id)=='pedigreeList') \{
         didx <- id$findex
         midx <- id$mindex
         n <- length(didx)
         \} 
     else \{
         n <- length(id)
         if (missing(dad.id) || length(dad.id) !=n)
             stop("Invalid father id")
         if (missing(mom.id) || length(mom.id) !=n)
             stop("Invalid mother id")
         midx <- match(mom.id, id, nomatch=0) # row number of my mom
         didx <- match(dad.id, id, nomatch=0) # row number of my dad
         \}
     if (n==1) return (0)  # special case of a single subject 
     parents <- which(midx==0 & didx==0)  #founders
 
     depth <- rep(0,n)
     # At each iteration below, all children of the current "parents" are
     #    labeled with depth 'i', and become the parents of the next iteration
     for (i in 1:n) \{
         child  <- match(midx, parents, nomatch=0) +
                   match(didx, parents, nomatch=0)
 
         if (all(child==0)) break
         if (i==n) 
             stop("Impossible pedegree: someone is their own ancestor")
 
         parents <- which(child>0) #next generation of parents
         depth[parents] <- i
         \}
     if (!align) return(depth)
     \nwhypf{kindepth-align1}{kindepth-align}{kindepth-align2}
 \}
\end{nwchunk}

The align argument is used only by the plotting routines.  
It makes the plotted result prettier in the following (fairly common)
case. 
Assume that subjects A and B marry, we have some ancestry information for
both, and that A's ancestors go back 3 generations, B's for only two.
If we add +1 to the depth of B and all her ancestors, then A and B
will be the same depth, and will plot on the same line.
Founders who marry in are also aligned.
However, if we have an inbred pedigree, there may not be a simple fix
of this sort.

The algorithm is
\begin{enumerate}
  \item First deal with founders.  If a particular founder marries in
    multiple times at multiple deaths (animal pedigrees), given that
    subject the min(depth of spouses).  These subjects cause trouble
    for the general algorithm below: the result would depend on the
    data set order.  Say A has offspring at level 2 and 0.  If ``2'' came
    first then a later step would pull the spouse at 0 down.
  \item Find any remaining mother-father pairs that are mismatched in depth.
    Deal with them one at a time.
    We think that aligning the top of a pedigree is more important
    than aligning at the bottom, so start with a mismatch pair of minimal
    depth.
  \item The children's depth is max(father, mother) +1.  Call the
    parent closest to the children ``good'' and the other ``bad''.
  \item  Chase up the good side, and get a list of all subjects connected
    to "good", including in-laws (spouse connections) and sibs that are
    at this level or above.  Call this agood (ancestors of good).
    We do not follow any connections at a depth lower than the 
    marriage in question, to get the highest marriages right.
    For the bad side, just get ancestors.
  \item Avoid pedigree loops!  If the agood list contains anyone in abad,
    then don't try to fix the alignment, otherwise:
    Push abad down, then run the pushdown algorithm to
    repair any descendents --- you may have pulled down a grandparent but
    not the sibs of that grandparent.
\end{enumerate}
    
It may be possible to do better alignment when the pedigree has loops,
but it is definitely beyond this program's abilities.  This could be
an addition to authint one day.
One particular case that we've seen was a pair of brothers that married
a pair of sisters.  Pulling one brother down fixes the other at the
same time.
The code below, however, says "loop! stay away!".
\begin{nwchunk}
\nwhypb{kindepth-align2}{kindepth-align}{kindepth-align1}=
 chaseup <- function(x, midx, didx) \{
     new <- c(midx[x], didx[x])  # mother and father
     new <- new[new>0]
     while (length(new) >1) \{
         x <- unique(c(x, new))
         new <- c(midx[new], didx[new])
         new <- new[new>0]
     \}
     x
 \}
             
 # First deal with any parents who are founders
 #  They all start with depth 0
 dads <- didx[midx>0 & didx>0]   # the father side of all spouse pairs
 moms <- midx[midx>0 & didx>0]
 founder <- (midx==0 & didx==0)
 if (any(founder[dads])) \{
     drow <- which(founder[dads])  # which pairs
     id  <- unique(dads[drow])     # id
     depth[id] <- tapply(depth[moms[drow]], dads[drow], min)
     dads <- dads[-drow]
     moms <- moms[-drow]
 \}
 if (any(founder[moms])) \{
     mrow <- which(founder[moms])  # which pairs
     id  <- unique(moms[mrow])     # id
     depth[id] <- tapply(depth[dads[mrow]], moms[mrow], min)
     dads <- dads[-mrow]
     moms <- moms[-mrow]
 \}
 
 # Get rid of duplicate pairs, which occur for any spouse with
 #  multiple offspring
 dups <- duplicated(dads + moms*n)
 if (any(dups)) \{
   dads <- dads[!dups]
   moms <- moms[!dups]
 \}
 
 npair<- length(dads)
 done <- rep(FALSE, npair)  #couples that are taken care of
 while (TRUE) \{
   pairs.to.fix <- which((depth[dads] != depth[moms]) & !done)
   if (length(pairs.to.fix) ==0) break
   temp <- pmax(depth[dads], depth[moms])[pairs.to.fix]
   who <- min(pairs.to.fix[temp==min(temp)])  # the chosen couple
   
   good <- moms[who]; bad <- dads[who]
   if (depth[dads[who]] > depth[moms[who]]) \{
     good <- dads[who]; bad <- moms[who]
   \}
   abad  <- chaseup(bad,  midx, didx)
   if (length(abad) ==1 && sum(c(dads,moms)==bad)==1) \{
     # simple case, a solitary marry-in
     depth[bad] <- depth[good]
   \}
   else \{
     agood <- chaseup(good, midx, didx)  #ancestors of the "good" side
     # For spouse chasing, I need to exclude the given pair
     tdad <- dads[-who]
     tmom <- moms[-who]
     while (1) \{
       # spouses of any on agood list
       spouse <- c(tmom[!is.na(match(tdad, agood))],
       tdad[!is.na(match(tmom, agood))])
       temp <- unique(c(agood, spouse))
       temp <- unique(chaseup(temp, midx, didx)) #parents
       kids <- (!is.na(match(midx, temp)) | !is.na(match(didx, temp)))
       temp <- unique(c(temp, (1:n)[kids & depth <= depth[good]]))
       if (length(temp) == length(agood)) break
       else agood <- temp
     \}
 
     if (all(match(abad, agood, nomatch=0) ==0)) \{
       # shift it down
       depth[abad] <- depth[abad] + (depth[good] - depth[bad])
       #
       # Siblings may have had children: make sure all kids are
       #   below their parents.  It's easiest to run through the
       #   whole tree
       for (i in 0:n) \{
         parents <- which(depth==i)
         child <- match(midx, parents, nomatch=0) +
             match(didx, parents, nomatch=0)
         if (all(child==0)) break
         depth[child>0] <- pmax(i+1, depth[child>0])
       \}
     \}
   \}
   # Once a subject has been shifted, we don't allow them to instigate
   #  yet another shift, possibly on another level
   done[dads==bad | moms==bad] <- TRUE
 \}
 if (all(depth>0)) stop("You found a bug in kindepth's alignment code!")
 depth
\end{nwchunk}

\subsection{familycheck}
The familycheck routine checks out a family id, by trying to construct its own
and comparing the results.
The input argument "newfam" is optional: if you've already created this
vector for other reasons, then putting the arg in saves time.


  If there are any joins, then an attribute "join" is attached.  It will be
   a matrix with famid as row labels, new-family-id as the columns, and
   the number of subjects as entries.  

\begin{nwchunk}
\nwhypn{familycheck}=
 # This routine checks out a family id, by trying to construct its own
 #  and comparing the results
 #
 # The input argument "newfam" is optional: if you've already created this
 #   vector for other reasons, then putting the arg in saves time.
 #
 # Output is a dataframe with columns:
 #   famid: the family id, as entered into the data set
 #   n    : number of subjects in the family
 #   unrelated: number of them that appear to be unrelated to anyone else 
 #          in the entire pedigree set.  This is usually marry-ins with no 
 #          children (in the pedigree), and if so are not a problem.
 #   split : number of unique "new" family ids.
 #            if this is 0, it means that no one in this "family" is related to
 #                   anyone else (not good)
 #            1 = everythings is fine
 #            2+= the family appears to be a set of disjoint trees.  Are you
 #                 missing some of the people?
 #   join : number of other families that had a unique famid, but are actually
 #            joined to this one.  0 is the hope.
 #
 #  If there are any joins, then an attribute "join" is attached.  It will be
 #   a matrix with famid as row labels, new-family-id as the columns, and
 #   the number of subjects as entries.  
 #
 familycheck <- function(famid, id, father.id, mother.id, newfam) \{
     if (is.numeric(famid) && any(is.na(famid)))
         stop ("Family id of missing not allowed")
     nfam <- length(unique(famid))
 
     if (missing(newfam)) newfam <- makefamid(id, father.id, mother.id)
     else if (length(newfam) != length(famid))
         stop("Invalid length for newfam")
 
     xtab <- table(famid, newfam)
     if (any(newfam==0)) \{
         unrelated <- xtab[,1]
         xtab <- xtab[,-1, drop=FALSE] 
         ## bug fix suggested by Amanda Blackford 6/2011
       \}
     else unrelated <-  rep(0, nfam)
 
     splits <- apply(xtab>0, 1, sum)
     joins  <- apply(xtab>0, 2, sum)
 
     temp <- apply((xtab>0) * outer(rep(1,nfam), joins-1), 1, sum)
 
     out <- data.frame(famid = dimnames(xtab)[[1]],
                       n = as.vector(table(famid)),
                       unrelated = as.vector(unrelated),
                       split = as.vector(splits),
                       join = temp,
                       row.names=1:nfam)
     if (any(joins >1)) \{
       tab1 <- xtab[temp>0,]  #families with multiple outcomes
       tab1 <- tab1[,apply(tab1>0,2,sum) >0] #only keep non-zero columns
       dimnames(tab1) <- list(dimnames(tab1)[[1]], NULL)
       attr(out, 'join') <- tab1
     \}
     
     out
   \}
 
 
\end{nwchunk}


\subsection{check.hint}
This routine tries to remove inconsistencies in spousal hints.
These and arise in autohint with complex pedigrees.
One can have ABA (subject A is on both the
left and the right of B), cycles, etc. 
Actually, these used to arise in autohint, I don't know if it's so
after the recent rewrite.
Users can introduce problems as well if they modify the hints.

\begin{nwchunk}
\nwhypn{check.hint}=
 check.hint <- function(hints, sex) \{
     if (is.null(hints$order)) stop("Missing order component")
     if (!is.numeric(hints$order)) stop("Invalid order component")
     n <- length(sex)
     if (length(hints$order) != n) stop("Wrong length for order component")
     
     spouse <- hints$spouse
     if (is.null(spouse)) hints
     else \{
         lspouse <- spouse[,1]
         rspouse <- spouse[,2]
         if (any(lspouse <1 | lspouse >n | rspouse <1 | rspouse > n))
             stop("Invalid spouse value")
         
         temp1 <- (sex[lspouse]== 'female' & sex[rspouse]=='male')
         temp2 <- (sex[rspouse]== 'female' & sex[lspouse]=='male')
         if (!all(temp1 | temp2))
             stop("A marriage is not male/female")
         
         hash <- n*pmax(lspouse, rspouse) + pmin(lspouse, rspouse)
         #Turn off this check for now - is set off if someone is married to two siblings
         #if (any(duplicated(hash))) stop("Duplicate marriage")
 
         # Break any loops: A left of B, B left of C, C left of A.
         #  Not yet done 
       \}
     hints
   \}
\end{nwchunk}
\end{document}
